\chapter{Hi, there!}

This is a beginner-level Python ``book''. It's closer to a series of
rambling notes at this point. The intent of the book is to teach you
basic Python, as well as establish an appropriate mathematical
background. This text focuses very little on Python-specific
techniques and ideas, and instead focuses on the mathematics and
problem-solving techniques used in programming.

Each chapter starts with a goal, and perhaps an outline of steps we'll
take to meet that goal. The chapters are designed so you can (ideally)
knock out one section in a reasonable sitting (an hour or two).

Note: if you're reading this in a PDF reader, you can click on
URLs, footnotes, citations, references to definitions, references to
figures, etc, and your PDF reader will take you there.\footnote{Hi,
  I'm a footnote!} There should be a ``back'' button in your PDF
reader, which you can click to go back to your previous spot.

The target audience is people with no background in programming, and
minimal background in pure mathematics.

In this first chapter, we're going to get you set up, and then write
some very simple programs in Python.

\subsection{CodingBat}

CodingBat (\url{http://codingbat.com/python}) has excellent Python
exercises, and it checks to make sure your solutions are correct. Many
of the exercises are just links to exercises on CodingBat.

\section{Getting set up}

You will want to install all of these things:

\begin{enumerate}
\item the Python 3 environment;
\item a plain-text editor;
\item if you're on Windows, Cygwin: \url{https://www.cygwin.com/}.
\end{enumerate}

If you're on a UNIX-like system, such as Linux, OS X, or a BSD system,
Python is probably already installed, and a text editor and terminal
emulator are almost certainly installed. The only hiccup might be the
version of Python.

On Windows, you'll want to download the latest Python 3 release from
here: \url{https://www.python.org/downloads/windows/}. (See
\cref{fig:python3-windows}.)

As for a ``plain-text editor'', that means a program to edit text
files that is critically \xtb{not a word processor}. There are
hundreds of text editors, and experienced programmers are usually
rather opinionated on their favorite text editor.

\begin{enumerate}
\item For the average new programmer, I'd recommend Atom:
  \url{https://atom.io/}. Atom is not a terrible choice for
  experienced developers, either.
\item If you are willing to learn how to use it, Vim is an excellent
  terminal-based editor: \url{http://www.vim.org/}. The
  poorly-designed website is not reflective of the quality of the text
  editor. Vim is probably the most popular text editor.
\item If you have several months to spare, and are already an
  experienced programmer, GNU Emacs is also very popular:
  \url{http://www.gnu.org/software/emacs/}. (This is what I use). In
  this case, the fast, and well-designed website is not reflective of
  the quality of the text editor.
\item If you are on Windows, I've heard good things about Notepad++:
  \url{https://notepad-plus-plus.org/}, although I've never tried it
  for more than an hour or two.
\end{enumerate}

\begin{figure}[ht]
  \centering
  \answergraph{images/python3-windows-gimped.png}
  \caption{Link you should click on for Python 3, iff you're using
    Windows.}
  \label{fig:python3-windows}
\end{figure}

Throughout this book, you'll need to run a lot of commands in the
terminal. On UNIX-like systems, you can open an application called
``Terminal,'' ``Terminal Emulator,'' or some variant thereof, and use
that. On Windows, use Cygwin. I will prefix commands you should run in
the terminal with a dollar sign. Expected output of the program
follows in subsequent lines.

\lstin{uname.txt}

\begin{remark}
  The red arrow just indicates that the previous line overflows onto
  the next line. It does not indicate an actual output character.
\end{remark}

\begin{remark}
  It's extremely important that you type in code examples and
  terminal commands yourself, and observe the results, rather than
  glancing at them in the book, or copying \& pasting.
\end{remark}

To test to see if Python 3 is installed on your system, first run
\code{which python} in a terminal.

\lstin{which-python.txt}

Then, run \code{python --version} to make sure it's using Python 3:

\lstin{python-version.txt}

On my system (Ubuntu 16.04), Python 2 is still the default. However,
the command \code{python3} gets me to Python 3.

\lstin{python3-version.txt}

The command to run Python 3 will vary from one system to another, so
therefore I will just say ``run the appropriate \code{python}
command,'' to refer to \code{python}, \code{python3}, or whatever it
is on your system.

If you run \code{python3} with no arguments, you'll get Python's
\term{Read/Eval/Print Loop}, or \term{REPL}.

\lstin{python3-repl.txt}

\begin{remark}
  The three `greater-than' signs are the standard Python REPL
  prompt. Therefore, if I want to indicate that you should type
  something into the Python REPL, I'll just prefix the line with
  \code{>>>}.
\end{remark}

At the REPL, you can type in a line of Python code. Python will
evaluate it and print the result. Then, under most circumstances,
Python will prompt you for another line of code.\footnote{The
  circumstance in which Python won't ask you for a new line of code is
  if you tell Python to quit via the \code{exit} command.} Examples:

\lstpy{hello-world-and-such.txt}

\begin{remark}
  The syntax highlighting is just there to help with readability; it
  doesn't actually mean anything.
\end{remark}

\subsection{Saving in files}

You can type entire programs into the REPL, and it will evaluate
them. However, you will usually want to save them in a file, so you
can use them later, and not have to type them out every time. Before
we do that, you need a small amount of background information.

I'd recommend creating a special directory for the programs in this
text. If you can't think of a name, I'd suggest \code{code}. To make a
directory, use the \code{mkdir} command:

\lstin{mkdir.txt}

There are a lot of UNIX commands, but here are the commands you'll
actually need to know as we move forward:

\begin{enumerate}
\item \code{mkdir} is for ``make directory.''
\item \code{cd} is for ``change directory.''
\item \code{pwd} is for ``print working directory.'' (Prints the name
  of the current directory to the console).
\item \code{ls} is for ``list.'' (Prints the contents of the current
  directory).
\item \code{cp} is for ``copy.''
\item \code{mv} is for ``move.'' (Or ``rename,'' if you prefer).
\item Most importantly, \code{man} is for ``manual.'' It will print
  out the manual for any command. I highly recommend glancing through
  the manual for all of the commands I just mentioned. For instance:

  \lstin{man-pwd.txt}

  To exit the manual, press \code{q}. The \code{#} indicates that the
  rest of the line is a \term{comment}, and is ignored by the shell
  (program that interprets \& executes terminal commands).

  One of the most important options for \code{man} is \code{man -k},
  which searches for manuals, also called \term{man pages}.

  \lstin{man-k.txt}
\end{enumerate}

\paragraph{Your first Python program run from a file.}

Now that you know the basics of UNIX, we're going to run a very small
program from a file. Python files typically have the \code{.py} file
extension. Here's the \code{hello world} program in a file:

\pyfile{code/hello_world.py}

Use your text editor to write\footnote{Yes, write it yourself, don't
  copy \& paste} that program to a file called \code{hello_world.py},
then run this command:

\lstin{python3-hello_world.txt}

\begin{remark}
  The triple-quoted string at the beginning of the file is a
  \term{block comment}. It serves to explain what the file does. You
  should get in the habit of putting a block comment at the top of all
  of your Python files explaining what the file does. In fact, later
  on, when we are writing functions, we'll write a block comment in
  each function explaining what the function does.
\end{remark}

\subsection{Exercises}

\begin{exercise}
  Write \& run programs that print out the following strings:

  \begin{enumerate}
  \item \code{The quick brown fox jumped over the lazy dog.}
  \item \code{Programming is not actually all that hard.}
  \item \code{Nixon did nothing wrong.}
  \end{enumerate}
\end{exercise}

\begin{exercise}
  Calculate the result of each of the following expressions using
  Python. (Doing this in the REPL is fine):

  \begin{enumerate}
  \item $3 \times 2$
  \item $11 \times 18$
  \item $\frac{541}{241}$
  \item $100 + 847 - 17$
  \end{enumerate}
\end{exercise}

