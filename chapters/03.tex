\chapter{The greatest common divisor}

Our goal at the end of this chapter is to write a function that
computes the greatest common divisor of two positive integers. I'll
try to develop the appropriate math background, and the appropriate
Python skills so that you'll understand what's going on at the end of
the chapter.

\paragraph{Why should I care about the greatest common divisor? That
  sounds like a really boring problem.}

One word: cryptography. Later in the book, I'll explain how some
modern cryptography systems work. It's not immediately clear right
now, but being able to quickly find the greatest common divisor of two
numbers is extremely important.

The na\"ive approach is to simply factor the two numbers, and then
pick the greatest common divisor. However, there is no known efficient
algorithm for factoring numbers. So, for very large numbers, this
quickly becomes unfeasible. Factoring is such a difficult problem,
that many modern cryptography protocols are designed around the fact
that factoring large numbers is extremely difficult.

There is an algorithm for efficiently computing the greatest common
divisor of two numbers, called the \term{Euclidean algorithm}. In this
chapter, I'll ``derive'' it, and prove that it indeed converges to the
greatest common divisor.

Here is the rough order in which I'm going to attack the problem:

\begin{enumerate}
\item explain sets and functions from a mathematical perspective,
\item define the natural numbers,
\item devise an algorithm for division with remainder, and finally
\item ``derive'' the Euclidean algorithm.
\end{enumerate}

\section{Sets}

Sets are kind of like lists. There are a couple of critical
differences:

\begin{enumerate}
\item in sets, the order in which elements appear does not matter;
\item in sets, the number of times an element appears does not matter.
\end{enumerate}

For small enough sets, you can list the elements in curly braces:

\begin{align*}
  \mset{1,2,3,4,5} & \text{ is a set;} \\
  \mset{3,2,5,1,4} & \text{ is the same set;} \\
  \mset{3,5,4,1,2,3,5} & \text{ is also the same set.}
\end{align*}

We can verify this in the REPL:

\lstpy{1to5-sets-lists.txt}

In Python, to say that $1$ is in the set $\mset{1,2,3,4,5}$, we write

\lstpy{1-in-1to5.txt}

In math, we write $$1 \in \mset{1,2,3,4,5}.$$

In math, we have the notions of \term{subsets} and
\term{supersets}. That is,

\begin{definition}
  $X$ is a subset of $Y$, written $X \subof Y$, if and only if
  \ctext{for all $x \in X$, $x \in Y$.}
\end{definition}

\begin{definition}
  $X$ is a proper subset of $Y$, written $X \psubof Y$, if and only if
  \ctext{for all $x \in X$, $x \in Y$;} and, there is some $y \in Y$
  such that $y \notin X$.
\end{definition}

The Python analogues for sets are \code{<=} and \code{<},
respectively.

\begin{remark}
  The symbol $\subset$ is more popular, but kind of ambiguous. It
  \emph{usually} refers to improper subsets. However, the symbols
  $\subof$ and $\psubof$ are unambiguous, so I'll use those.
\end{remark}

For instance, $$\mset{2,3,4} \subof \mset{1,2,3,4,5}.$$ In Python,
this is

\lstpy{subof.txt}

\begin{remark}
  Sets in math are actually much more closely related to \emph{types},
  rather than lists. Why this is the case is largely beyond the scope
  of this book. It is worth mentioning, though.
\end{remark}

\term{Supersets} are defined analogously:

\begin{definition}
  $X$ is a superset of $Y$, written $$X \supof Y,$$ if and only if
  $Y \subof X$.
\end{definition}

\begin{definition}
  $X$ is a proper superset of $Y$, written $$X \psupof Y,$$ if and
  only if $Y \psubof X$.
\end{definition}

Set equality is defined as two sets containing precisely the same
elements. In other words,

\begin{definition}
  Given two sets $X$ and $Y$, $X$ is equal to $Y$, written $$X = Y,$$
  if and only if \ctext{$X \subof Y$ and $X \supof Y$.}
\end{definition}

Much like lists, there is a set containing no elements, called the
\term{null set}. The symbol varies from book to book, but it's usually
one of $\emptyset$, $\varnothing$, $\phi$, or some variant
thereof. I'm going to use $\emptyset$, because it's the most standard.

\subsection{Combining sets}

There are three common ways to combine sets to create new sets:

\begin{enumerate}
\item the \term{union} of two sets, where you take elements that are
  in either set (or both);
\item the \term{intersection} of two sets, where you take elements
  that are in both sets;
\item the \term{difference} of two sets, where you take elements that
  are in one set, but not the other.
\end{enumerate}

\section{Mathematical background}

First of all, what are the ``positive integers''? \term{Integers} are
whole numbers: $$\Z = \mset{0, \pm 1, \pm 2, \pm 3, \dots}.$$ The
\term{natural numbers} are the positive integers:
$$\N = \mset{1,2,3,\dots}.$$

\begin{remark}
  Some people use $0$ as the first natural number. It doesn't matter a
  whole not. In our case, it's more convenient to have $1$ as the
  first natural number.
\end{remark}

\subsection{Sets}

Before we go much further, I'd like to introduce the concept of a
\term{set}. A set is a collection of objects, called \term{elements},
with no notion of order or multiplicity. Finite sets with few elements
are typically denoted with braces: \ctext{$\mset{1,2,3,4}$ is a set.}
The things distinguishing sets from \term{lists} or \term{vectors}
are:

\begin{enumerate}
\item in a set, the order in which elements appear is irrelevant;
\item in a set, the number of times that an element appears is
  irrelevant.
\end{enumerate}

Therefore, all of the following are the same set:

\begin{align*}
  & \mset{1,2,3,4} \\
  = & \mset{1,1,2,2,3,3,4,4} && \text{(Each element is duplicated).} \\
  = & \mset{1, 3, 3, 4, 1, 4, 2} && \text{(Different order, and some duplication).}
\end{align*}

Given a set $A$, and an element $x$, to say \ctext{$x$ is an element
  of $A$,} we write $$x \in A.$$ If \ctext{$x$ is not an element of
  $A$,} then we write $$x \notin A.$$

\paragraph{Set comprehension notation.}

This notation is common:

\begin{equation*}
  \scomp{\text{what each element looks like}}{\text{conditions about
      each element}}
\end{equation*}

In Python, we have list comprehensions, which are similar:

\lstpy{comprehensions.txt}

\subsection{Functions}

A \term{mathematical function} takes elements from one set and maps
them to elements in another set. For instance,

\begin{align*}
  f & : \N \to \N \\
  f(x) & = x + 3
\end{align*}

Python has things called ``functions'', but they are slightly
different. Here's how we would define that function in Python:

\pyfile{code/x_plus_3.py}

The difference comes down to a property called \term{referential
  transparency}:

\begin{axiom}
  \label{foo}
  If $f : A \to B$, and $x, y \in A$, then, \ctext{if $x = y$, then
    $f(x) = f(y)$.}
\end{axiom}

Functions in Python do not usually have this property. For instance,

\lstpy{random-randint.txt}

I called the same ``function'' with the same arguments, and got a
different result. That cannot happen in a mathematical function.

The following definitions are useful computations that we'll be using:

\begin{definition}
  The \term{Cartesian product} of two sets $A$ and $B$ is the set of
  ordered pairs where the first value is in $A$, and the second value
  is in $B$. That is,
  $$A \times B = \scomp{(a, b)}{a \in A, \text{ and } b \in B}.$$
\end{definition}

\begin{definition}
    
\end{definition}

\subsection{Division with remainder}

This is the way you learned to divide in elementary school. We want a
mathematical function, that, given two positive integers $a, b \in
\N$, returns the \term{quotient} of $a$ and $b$, plus a
\term{remainder}, which might be $0$.

\begin{lemma}
  Given two positive integers $a, b \in \N$, with $a \ge b$ we can
  always write $$a = qb + r,$$ where $q \in \N$ is the
  \term{quotient}, and $r$ is the \term{remainder}. The remainder can
  be zero, but it must be less than $b$. Therefore,
  $$r \in \scomp{x \in \Z}{0 \le x < b}.$$
\end{lemma}


See \cref{foo}.