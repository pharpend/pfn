\chapter{The greatest common divisor}

Our goal at the end of this chapter is to write a function that
computes the greatest common divisor of two positive integers. I'll
try to develop the appropriate math background, and the appropriate
Python skills so that you'll understand what's going on at the end of
the chapter.

\section{Mathematical background}

First of all, what are the ``positive integers''? \term{Integers} are
whole numbers: $$\Z = \mset{0, \pm 1, \pm 2, \pm 3, \dots}.$$ The
\term{natural numbers} are the positive integers:
$$\N = \mset{1,2,3,\dots}.$$

\begin{remark}
  Some people use $0$ as the first natural number. It doesn't matter a
  whole not. In our case, it's more convenient to have $1$ as the
  first natural number.
\end{remark}

\subsection{Sets}

Before we go much further, I'd like to introduce the concept of a
\term{set}. A set is a collection of objects, called \term{elements},
with no notion of order or multiplicity. Finite sets with few elements
are typically denoted with braces: \ctext{$\mset{1,2,3,4}$ is a set.}
The things distinguishing sets from \term{lists} or \term{vectors}
are:

\begin{enumerate}
\item in a set, the order in which elements appear is irrelevant;
\item in a set, the number of times that an element appears is
  irrelevant.
\end{enumerate}

Therefore, all of the following are the same set:

\begin{align*}
  & \mset{1,2,3,4} \\
  = & \mset{1,1,2,2,3,3,4,4} && \text{(Each element is duplicated).} \\
  = & \mset{1, 3, 3, 4, 1, 4, 2} && \text{(Different order, and some duplication).}
\end{align*}

Given a set $A$, and an element $x$, to say \ctext{$x$ is an element
  of $A$,} we write $$x \in A.$$ If \ctext{$x$ is not an element of
  $A$,} then we write $$x \notin A.$$

\paragraph{Set comprehension notation.}

This notation is common:

\begin{equation*}
  \scomp{\text{what each element looks like}}{\text{conditions about
      each element}}
\end{equation*}

In Python, we have list comprehensions, which are similar:

\lstpy{comprehensions.txt}

\subsection{Functions}

A \term{mathematical function} takes elements from one set and maps
them to elements in another set. For instance,

\begin{align*}
  f & : \N \to \N \\
  f(x) & = x + 3
\end{align*}

Python has things called ``functions'', but they are slightly
different. Here's how we would define that function in Python:

\pyfile{code/x_plus_3.py}

The difference comes down to a property called \term{referential
  transparency}:

\begin{axiom}
  If $f : A \to B$, and $x, y \in A$, then, \ctext{if $x = y$, then
    $f(x) = f(y)$.}
\end{axiom}

Functions in Python do not usually have this property. For instance,

\lstpy{random-randint.txt}

I called the same ``function'' with the same arguments, and got a
different result. That cannot happen in a mathematical function.

The following definitions are useful computations that we'll be using:

\begin{definition}
  The \term{Cartesian product} of two sets $A$ and $B$ is the set of
  ordered pairs where the first value is in $A$, and the second value
  is in $B$. That is,
  $$A \times B = \scomp{(a, b)}{a \in A, \text{ and } b \in B}.$$
\end{definition}

\begin{definition}
    
\end{definition}

\subsection{Division with remainder}

This is the way you learned to divide in elementary school. We want a
mathematical function, that, given two positive integers $a, b \in
\N$, returns the \term{quotient} of $a$ and $b$, plus a
\term{remainder}, which might be $0$.

\begin{lemma}
  Given two positive integers $a, b \in \N$, with $a \ge b$ we can
  always write $$a = qb + r,$$ where $q \in \N$ is the
  \term{quotient}, and $r$ is the \term{remainder}. The remainder can
  be zero, but it must be less than $b$. Therefore,
  $$r \in \scomp{x \in \Z}{0 \le x < b}.$$
\end{lemma}
