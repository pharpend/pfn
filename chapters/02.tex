\chapter{Basics}

Our goal at the end of this chapter is to write a program that does
the following things, in a loop:

\begin{enumerate}
\item Read a piece of text from the user.
\item Print out the text but reversed.
\item If the user inputs `exit', then the program exits, but before
  we print out the reversed text.
\end{enumerate}

Vocabulary term: a piece of text is called a \term{string}.

To print out the text, use the \code{print()} function. To read text
input, we use the \code{input()} function. Examples:

\lstpy{input-example.txt}

Knowing about the function \code{input()} is a piece of trivia, that,
in the grand scheme of things, is unimportant. However, being able to
apply the thought process and techniques in this section to similar
problems is very important.

The broadest technique in programming is to break a problem down into
smaller problems. Here is how I would break down this problem:

\begin{enumerate}
\item start up the program;
\item read a string from the user using \code{input()};
\item if the string is \code{exit}, then die;
\item else, reverse the string;
\item print the reversed string;
\item go back to 2.
\end{enumerate}

There are four important concepts in my breakdown:

\begin{enumerate}
\item the \term{variable}, i.e. storing and changing a value;
\item the \term{conditional}, i.e. if A then B else C;
\item the \term{loop}, i.e. ``go back to 2;''
\item a \term{subroutine}, where we delegate repetitive tasks to
  smaller programs stored inside our main program.
\end{enumerate}

The latter three concepts are examples of \term{flow control
  structures}. We'll learn more about them in this
chapter. Conditional statements are broad enough that they require
their own section.

\section{Variables}

The first thing is a variable. Most people are familiar with the
concept of a variable from mathematics courses. In most programming
languages, a variable can actually vary. So, for instance, in math,
this is (more or less) nonsense:

\begin{align*}
  a & = 2 \\
  a & = a + 10 \\
  a & \text{ is now equal to 12.}
\end{align*}

At the very least, it's rather uncommon. However, in most programming
languages, it's extremely common. It's the most fundamental and common
thing you do. Here's how we'd represent that in Python:

\lstpy{python-var-1.txt}

Every variable (indeed, every value) has a \term{type}. To get the
type of an object in Python, you use the \code{type()} function:

\lstpy{python-types-1.txt}

\begin{remark}
  In Python, types are less important, because Python uses \term{weak
    typing}, and \term{dynamic typing}, where Python does not actually
  determine the type of each object until it actually runs the
  code. However, in certain programming languages (languages with
  \term{static typing} and \term{strong typing}), types are extremely
  important.

  Weak typing is a fundamental weakness of Python, and indeed all
  weakly typed languages, including Ruby. Small mental errors can
  translate into impossible-to-find bugs in your code, that may take
  several days to find, let alone fix. This is why I strongly prefer
  static and strongly typed languages, such as Haskell. In Haskell,
  most bugs are caught before you even compile the software.

  If you take strong typing to the extreme, you get languages like
  Idris and Coq, where the type system is strong enough that you can
  actually mathematically prove the correctness of your program with
  the type system.
\end{remark}

\begin{remark}
  Fair warning: my experience is predominantly in \term{functional}
  programming languages, which do not have the concept of varying
  variables. Therefore, much of the code I write will not use the
  ``Pythonic'' method, and will instead employ a more Haskell-like
  approach.
\end{remark}

\subsection{Legal variable names}

Python, and indeed every language, has rules about legal variable
names:

\begin{enumerate}
\item variable names \emph{must} start with a letter or an underscore;
\item variable names \emph{should} start with a lowercase letter or an
  underscore;
\item variable names may only contain letters, numbers, and underscores;
\item variable names are case-sensitive;
\item separate words in variables should use the \code{snake_case}
  convention (as opposed to the \code{idontknowwhatwordsare} or
  \code{camelCase} conventions).
\end{enumerate}

\begin{exercise}
  What do you suppose the types of these values are? Check your
  guesses in the REPL:

  \begin{enumerate}
  \item \code{2 + 2}
  \item \code{28 / 13}
  \item \code{28 / 14}
  \item \code{'hello, my name is joe'}
  \end{enumerate}
\end{exercise}

\begin{exercise}
  What do you think the following program does?

  \pyfile{code/print_plus_3.py}

  \xtb{Hint}: \emph{\code{int(x)} takes any value and tries to convert
    it to an integer. It might be worth toying around with
    \code{int()} in the REPL.}

  Try it and see.
\end{exercise}

\begin{exercise}
  Which of the following are legal variable names?

  \begin{enumerate}
  \item \code{Hello_my_name_is_Peter!}
  \item \code{Hello_my_name_is_Peter}
  \item \code{one+one_equals_2}
  \item \code{one_plus_one_equals_2}
  \item \code{n00b}
  \item \code{0perator}
  \end{enumerate}

  Try these out in the REPL (assign each name the value of \code{0} or
  something).
\end{exercise}

\section{Conditionals and Boolean algebra}

Conditional statements, i.e. if/then/else, are based on
\term{Booleans}\footnote{``Boolean'' is capitalized because it is
  named after an English mathematician, George Boole.}.

\begin{definition}
  A Boolean is either true or false.
\end{definition}

The study and manipulation of Booleans is called \term{Boolean
  algebra}. Let's explore Booleans in Python:

\lstpy{python-bools-1.txt}

In our goal, we want to be able to create a Boolean that tells us
whether or not the string is `exit'.

\subsection{Combining Booleans}

Booleans are kind of useless unless you can

\begin{enumerate}
\item combine them to make new Booleans, and
\item use them to determine whether or not to do something.
\end{enumerate}

There are three common ways to combine Booleans:

\begin{enumerate}
\item negation, where \code{not True = False}, and \code{not False =
    True};
\item an ``and'' statement, such as \code{a and b}, which is true if
  and only if both \code{a} and \code{b} are true;
\item an ``or'' statement, such as \code{a or b}, which is true if and
  only if at least one of \code{a} and \code{b} is true.
\end{enumerate}

\lstpy{python-bools-1.txt}

A common exercise is to build a \term{truth table} for the
operations. I've already given you some of the answers with my example

\begin{table}
  \centering
  \begin{tabu}{r|c|c}
    \code{and} & \code{True} & \code{False} \\
    \tabucline \\
    \code{True} & \code{True} & \code{False} \\
    \tabucline \\
    \code{False} & & \\
  \end{tabu}
  \caption{Partially filled in truth table for \code{and}.}
  \label{tbl:truth-table-and}
\end{table}

\begin{table}
  \centering
  \begin{tabu}{r|c|c}
    \code{or} & \code{True} & \code{False} \\
    \tabucline \\
    \code{True} & & \code{True} \\
    \tabucline \\
    \code{False} & & \code{False} \\
  \end{tabu}
  \caption{Partially filled in truth table for \code{or}.}
  \label{tbl:truth-table-or}
\end{table}

The next thing is negation:

\lstpy{python-negation-1.txt}

This immediately gives rise to the following lemma, which is obvious:

\begin{lemma}
  \label{thm:double-neg-bool}
  For all Booleans \code{a}, \code{not (not a)} is equal to \code{a}.
\end{lemma}

It is not all that interesting. However, the following properties,
called \term{De Morgan's laws}, are:

\begin{lemma}
  \label{thm:de-morgan-1}
  For all Booleans \code{a} and \code{b}, \code{not (a and b)} is
  equal to \code{(not a) or (not b)}.
\end{lemma}

Think about it:

\begin{enumerate}
\item I told you that I went to the park and the grocery store;
\item but you know that I didn't go to the park, or that I didn't go
  to the grocery store;
\item then I lied to you.
\end{enumerate}

The converse is also true:

\begin{enumerate}
\item I told you that I didn't go to the park or I didn't go to the grocery store;
\item but you know that I did in fact go to the park, and that I did
  go to the grocery store;
\item then I lied to you.
\end{enumerate}

It might help to compute a few of these values in the REPL, or just
make a truth table (\cref{exc:de-morgan-table-1}).

This corollary can be obtained by combining \cref{thm:double-neg-bool}
and \cref{thm:de-morgan-1}:

\begin{corollary}
  \label{thm:de-morgan-2}
  For all Booleans \code{a} and \code{b}, \code{not (a or b)} is equal
  to \code{(not a) or (not b)}.
\end{corollary}

If you're not convinced, think about it for a second, or build a truth
table (\cref{exc:de-morgan-table-2}).

Next, we have the \term{distributive laws}:

\begin{lemma}
  For all Booleans \code{a}, \code{b}, and \code{c}, \code{a and (b or
    c)} is equal to \code{(a and b) or (a and c)}.
\end{lemma}

This is similar to the distributive law in math, for real
numbers: $$a(b + c) = ab + ac.$$ This is also true:

\begin{lemma}
  For all Booleans \code{a}, \code{b}, and \code{c}, \code{a or (b and
    c)} is equal to \code{(a or b) and (a or c)}.
\end{lemma}

The analog doesn't hold in math: $$a + bc \ne (a + b)(b + c).$$

\subsection{CodingBat}

I mentioned at the beginning of the book that many of the exercises
are just to complete CodingBat exercises. This is because CodingBat
will check your answers for you, and give you immediate feedback. You
have \emph{almost} enough information to do the first one. There's one
critical piece of information missing, and that's the notion of a
\term{function}.

The syntax for functions is a bit difficult to explain. It will be
easier if I just write a few functions, then let you figure things
out.  To illustrate, I'll do the first couple of CodingBat exercises,
then leave others in the exercises.

\begin{example}
  (This is from \url{http://codingbat.com/prob/p173401}).

  We are given two variables: \code{weekday} and \code{vacation}, both
  of which are Booleans. We want to write a function that determines
  whether or not we sleep in. If it is the weekend (i.e. \code{not
    weekday}), or we are on vacation, then we sleep in. Else, we don't

  \begin{solution}
    CodingBat gives you the beginning of the function:

\begin{lstlisting}[language=Python]
def sleep_in(weekday, vacation):
    # your code goes here
\end{lstlisting}
    You need to write something in the body that, at the end, returns
    a Boolean determining whether or not we sleep in.

    We combine these two conditions with \code{or}

    \begin{enumerate}
    \item it's the weekend (i.e \code{not weekday});
    \item we're on vacation (i.e. \code{vacation}).
    \end{enumerate}

    \pyfile{code/codingbat/warmup1/sleep_in.py}

    This gives the correct result (see \cref{fig:sleep_in}).
  \end{solution}
\end{example}

\begin{figure}[h]
  \centering
  \answergraph{images/sleep_in.png}
  \caption{The result of our \code{sleep_in()} solution.}
  \label{fig:sleep_in}
\end{figure}

\begin{example}
  (From \url{http://codingbat.com/prob/p120546}).

  \begin{quote}
    We have two monkeys, a and b, and the parameters \code{a_smile}
    and \code{b_smile} indicate if each is smiling. We are in trouble
    if they are both smiling or if neither of them is smiling. Return
    \code{True} if we are in trouble.
  \end{quote}

  \begin{solution}
    There are two things, which can be combined with \code{or}:

    \begin{enumerate}
    \item both monkeys smiling (\code{a_smile and b_smile});
    \item neither monkey smiling (\code{(not a_smile) and (not
        b_smile)}).
    \end{enumerate}

    Therefore, our result is

    \pyfile{code/codingbat/warmup1/monkey_trouble.py}

    As you can see in \cref{fig:monkey_trouble}, this works!
  \end{solution}
\end{example}

\begin{figure}[h]
  \centering
  \answergraph{images/monkey_trouble.png}
  \caption{The result of running our \code{monkey_trouble()} solution
    in CodingBat.}
  \label{fig:monkey_trouble}
\end{figure}

There's more to cover, but this section is too long as it is, so here
are some exercises:

\subsection{Exercises}

\begin{exercise}
  Fill in \cref{tbl:truth-table-and}.
\end{exercise}
\begin{exercise}
  Fill in \cref{tbl:truth-table-or}.
\end{exercise}
\begin{exercise}
  \label{exc:and-comm}
  Do you think the following property is true?

  \emph{For all Booleans \code{a} and \code{b}, \code{a and b} equals
    \code{b and a}.}

  Check your truth table to see.
\end{exercise}
\begin{exercise}
  Do the same as \cref{exc:and-comm}, but for \code{or}.
\end{exercise}
\begin{exercise}
  \label{exc:and-assoc}
  Do you think the following property is true?

  \emph{For all Booleans \code{a}, \code{b}, and \code{c} \code{a and
      (b and c)} equals
    \code{(a and b) and c)}.}

  Constructing a truth table here would require 3 dimensions, so just
  convince yourself whether or not it's true, maybe trying things in
  the REPL.
\end{exercise}
\begin{exercise}
  Do the same as \cref{exc:and-assoc}, but for \code{or}.
\end{exercise}
\begin{exercise}
  \label{exc:de-morgan-table-1}
  Build two truth tables for \cref{thm:de-morgan-1}. That is, build
  one for \code{not (a and b)}, then build a separate truth table for
  \code{(not b) or (not a)}, and then verify that they are identical.

  \begin{center}
    \begin{tabu}{r|c|c}
      \code{not (a and b)} & \code{True} & \code{False} \\
      \tabucline \\
      \code{True} && \\
      \tabucline \\
      \code{False} && \\
    \end{tabu}
  \end{center}
  \begin{center}
    \centering
    \begin{tabu}{r|c|c}
      \code{(not a) or (not b)} & \code{True} & \code{False} \\
      \tabucline \\
      \code{True} && \\
      \tabucline \\
      \code{False} && \\
    \end{tabu}
  \end{center}
\end{exercise}
\begin{exercise}
  \label{exc:de-morgan-table-2}
  Build two truth tables for \cref{thm:de-morgan-2}, in similar fashion
  to \cref{exc:de-morgan-table-1}:

  \begin{center}
    \centering
    \begin{tabu}{r|c|c}
      \code{not (a or b)} & \code{True} & \code{False} \\
      \tabucline \\
      \code{True} && \\
      \tabucline \\
      \code{False} && \\
    \end{tabu}
  \end{center}

  \begin{center}
    \centering
    \begin{tabu}{r|c|c}
      \code{(not a) and (not b)} & \code{True} & \code{False} \\
      \tabucline \\
      \code{True} && \\
      \tabucline \\
      \code{False} && \\
    \end{tabu}
  \end{center}
\end{exercise}

\section{Numerical tests and some mathematical language}

In the lemmas in the previous section, it's awfully tiring to write
``is equal to'' every time I want to mention that two things are
equal. Instead, Python has the \code{==} operator:

\lstpy{python-equals-equals.txt}

\begin{remark}
  It's important to note that \code{a == b} \emph{tests} to see if
  \code{a} and \code{b} are equal, whereas \code{a = b} \emph{assigns}
  \code{a} to whatever \code{b} is.
\end{remark}

The analog of $\ne$ is \code{!=}:

\begin{lstlisting}[language=Python]
>>> 2 != 4
True
>>> 2 != 2
False
\end{lstlisting}

In fact, most of the mathematical operators have analogs in
Python. \Cref{tbl:math-python} might be helpful.

\begin{table}
  \centering
  \begin{tabu}{r|l}
    \textbf{Math} & \textbf{Python} \\
    \tabucline \\
    $a := b$ & \code{a = b} \\
    $a = b$ & \code{a == b} \\
    $a \ne b$ & \code{a != b} \\
    $a > b$ & \code{a > b} \\
    $a \ge b$ & \code{a >= b} \\
    $a < b$ & \code{a < b} \\
    $a \le b$ & \code{a <= b} \\
  \end{tabu}
  \caption{Mathematical statements, and their equivalent in Python}
  \label{tbl:math-python}
\end{table}

\subsection{Some number $x$ is within $\epsilon$ of $y$}

This is a problem that comes up a lot in programming. For instance, if
you're writing a shoot-em-up game, you want to be able to tell if a
bullet hit its target. It's really hard to properly detect collisions,
like you would in real life. Instead, what you do is see if the bullet
was ever sufficiently close to its target that you can say ``that's
good enough, it hit the target''.

An example that complicated won't come up for a while. However, this
is one of the CodingBat problems:

\begin{quotation}
  Given an int $n$, return \True if it is within 10 of 100 or
  200. Note: \code{abs(num)} computes the absolute value of a number.
\end{quotation}

In that case, you want to be able to tell if some number $n$ is within
$10$ of $100$, and again check if it's within $10$ of $200$. A na\"ive
approach is this:

\begin{enumerate}
\item check if $n \ge 90$, and $n \le 110$, then
\item check if $n \ge 190$, and $n \le 210$.
\end{enumerate}

That quickly becomes cumbersome to read. Instead, take a page out of a
topology textbook: the distance between two points $x$ and $y$ is
equal to $\abs{x - y}$. Note: here, $\abs{x}$ denotes the
``magnitude,'' or ``norm'' of $x$. With ordinary numbers, $\abs{x}$ is
just the absolute value:

\begin{definition}
  For a real number $x$,

  \begin{equation}
    \label{eq:abs-val}
    \abs{x} = \left\{
      \begin{split}
        x & \text{ if } x \ge 0 \\
        -x & \text{ if } x < 0
      \end{split}
    \right.
  \end{equation}
\end{definition}

Note this corollary

\begin{corollary}
  For all real numbers $x$, $$\norm{x} = \norm{-x}.$$
\end{corollary}

This implies

\begin{corollary}
  For all real numbers $x$ and $y$, $$\norm{x - y} = \norm{y - x}.$$
\end{corollary}

Therefore, \ctext{$\norm{100 - x} \le 10$ if and only if
  $90 \le x \le 110$.}  Likewise, \ctext{$\norm{200 - x} \le 10,$ if
  and only if $190 \le x \le 210.$} Therefore, in this case, you can
simplify the problem by combining these with \code{or}:

\begin{enumerate}
\item check if $\norm{x - 100} \le 10$, using \code{abs()}
\item check if $\norm{x - 200} \le 10$, using \code{abs()}
\end{enumerate}

I'm starting to sound less like a normal person, and more like a
mathematician. Therefore, there are two routes:

\begin{enumerate}
\item ``dumb down'' my language so non-mathematicians can understand
  it, or
\item teach you what all this terminology is.
\end{enumerate}

I vote for \#2, because the mathematical methods of thinking are very
helpful. The next chapter will go over much of the mathematical
terminology that you'll need. For now, I'll give you the bare minimum
to do some of the CodingBat problems.

\subsection{Exercises}

These are the problems I could find that use the methods of this
section, but not those in the next section.

\begin{exercise}
  Use \cref{tbl:math-python} and the methods described in this section
  to do \url{http://codingbat.com/prob/p124676}. (\code{near_hundred})
\end{exercise}
\cbex{http://codingbat.com/prob/p124984}{makes10}.
\cbex{http://codingbat.com/prob/p166884}{parrot_trouble}

\section{If/then/Else}

\subsection{If/then/else}

Now that you understand what Booleans are, and how to perform simple
numerical tests, we need to be able to instruct Python to

\begin{enumerate}
\item given a Boolean \code{b};
\item if \code{b} is true, do something;
\item if \code{b} is false, do something else.
\end{enumerate}

In Python, this is achieved via the \code{if} clause. The general
syntax is

\begin{lstlisting}[language=Python]
if b:
    # do something here
else:
    # do something different
\end{lstlisting}

The \code{#} represents a comment, which is ignored.

\begin{remark}
  The above is not valid code, because you'd need actual code inside
  the \code{if} and \code{else} blocks, rather than just a comment.
\end{remark}

\begin{remark}
  Please note the trailing colons after \code{if b} and
  \code{else}. It's really easy to forget them.
\end{remark}

\paragraph{Indentation.}

Nobody can agree on good indentation practices. However, Python has a
very extensive style guide, called Python Enhancement Protocol 8, or
PEP 8: \url{https://www.python.org/dev/peps/pep-0008/}.

Here are the typical spats:

\begin{enumerate}
\item How many spaces per indentation level? It's usually either 2 or
  4, but sometimes 8.
\item Should you use spaces or tabulator characters? Nowadays, spaces
  are preferred, because everyone agrees how wide a space is. Nobody
  can agree how wide a tab is (see \#1).
\end{enumerate}

PEP 8 states the following:

\begin{quotation}
  Use 4 spaces per indentation level. 

  \dots

  Spaces are the preferred indentation method.

  Tabs should be used solely to remain consistent with code that is already indented with tabs.

  Python 3 disallows mixing the use of tabs and spaces for indentation.
\end{quotation}

Almost everyone agrees that mixing tabs and spaces is evil. This book
will follow PEP 8's suggestion and use 4 spaces.

\paragraph{Back to work.}

Let's look at a simple program that takes in two numbers from the
user, and checks if their sum is $20$.

\pyfile{code/sum_is_twenty.py}

Here's an example input:

\lstin{sum_is_twenty.txt}

Often times, if the first condition fails to be true, you'll want to
test for other conditions inside the \code{else} block:

\begin{lstlisting}[language=Python]
if b1:
    # do something here
else:
    # more conditions
    if b2:
        # do something here
    else:
        # more conditions
      if b3:
          # do something here
      else:
          # ...
\end{lstlisting}

This situation is rather common, so instead we have \code{elif}:

\begin{lstlisting}[language=Python]
if b1:
    # do something here
elif b2:
    # do something here
elif b3:
    # do something here
# ...
else:
    # finally, if everything fails, here
\end{lstlisting}

I'll do one of the CodingBat problems, and leave the rest to you.

\begin{example}
  (This is \url{http://codingbat.com/prob/p162058}).

  \begin{quotation}
    Given 2 int values, return True if one is negative and one is
    positive. Except if the parameter "negative" is True, then return
    True only if both are negative.
  \end{quotation}

  \begin{solution}
    That word salad might be tough to parse. I'll try to break it down
    into a nice pretty tree:

    Given two numbers $a$ and $b$

    \begin{enumerate}
    \item If \code{negative} is \True ,

      then, return \True if both $a, b < 0$.
    \item Else,

      return \True\ if $a < 0 < b$ or $b < 0 < a$ (i.e. one is
      positive and the other is negative).
    \end{enumerate}

    There's a handy mathematical trick, here, which is, for two
    numbers $a$ and $b$:

    \begin{align*}
      ab > 0 & \text{ if $a$ and $b$ have the same sign;} \\
      ab < 0 & \text{ if $a$ and $b$ have opposite signs;} \\
      ab = 0 & \text{ if at least one of $a$ and $b$ is zero.}
    \end{align*}

    Therefore, to test if the two numbers have opposite signs, we
    simply test for $ab < 0$.

    Here's my solution:

    \pyfile{code/codingbat/warmup1/pos_neg.py}

    It gives the correct result (see \cref{fig:pos_neg}).
  \end{solution}
\end{example}

\begin{figure}[h]
  \centering
  \answergraph{images/pos_neg.png}
  \caption{Result of my solution for \code{pos_neg()}.}
  \label{fig:pos_neg}
\end{figure}

\subsection{Exercises}

\begin{exercise}
  Construct two truth tables to convince yourself that these two
  things are the same.

  \lstpy{if-x-return-true.py}

  \begin{center}
    \begin{tabu}{r|c|c}
      \code{if/else} & \code{True} & \code{False} \\
      \tabucline \\
      \True && \\
      \tabucline \\
      \False && \\
    \end{tabu}
  \end{center}

  \lstpy{return-some-boolean.py}

  \begin{center}
    \begin{tabu}{r|c|c}
      \code{return} & \code{True} & \code{False} \\
      \tabucline \\
      \True && \\
      \tabucline \\
      \False && \\
    \end{tabu}
  \end{center}

  I put this exercise first to suggest a way to eliminate a common
  (redundant) idiom in your code, and make your code a bit cleaner.
\end{exercise}

\begin{exercise}
  Here's another common redundant idiom:

  Given a Boolean \code{x}, write out a truth table to convince
  yourself that \code{x} always has the same value as \code{x ==
    True}.

  \begin{center}
    \begin{tabu}{r|l}
      \code{x} & \code{x == True} \\
      \tabucline \\
      \True & \\
      \False & \\
    \end{tabu}
  \end{center}

  Therefore, you should never write something like \code{if x ==
    True}. Instead replace it with just \code{if x}.
\end{exercise}

\begin{exercise}
  Use the \code{abs()} function to do
  \url{http://codingbat.com/prob/p197466}. (\code{diff21})
\end{exercise}

\begin{exercise}
  Use \cref{tbl:math-python}, as well as your reasoning ability to do
  \url{http://codingbat.com/prob/p162058}. (\code{pos_neg})
\end{exercise}

\begin{exercise}
  Write a program that takes a number (via \code{input()}), and prints
  out whether or not it's greater than $34$.
\end{exercise}

\cbex{http://codingbat.com/prob/p141905}{sum_double}

\begin{exercise}
  Do all of Logic-1: \url{http://codingbat.com/python/Logic-1}.
\end{exercise}

\section{Lists and Strings}

The next important type is a \term{list}. Informally, you can think of
a string as a list of characters.\footnote{This is a rather
  inefficient implementation, but it's a decent way to think about
  strings. Nonetheless, it is used in a few languages.}

The syntax for lists is to surround them in square brackets, with the
elements separated by commas:

\lstin{lists-01.txt}

To get the $i$th element in a list $x$, use \code{x[i]}:

\lstin{lists-02.txt}

\xtb{Important:} list indices start at $0$. This is the case in most
languages.

In Python, you can use negative indices, starting with $-1$, to pick
elements from the end of the list:

\lstin{lists-03.txt}

To get the $i$th element, up to (but not including) the $j$th element,
with $i < j$, use \code{x[i:j]}:

\lstin{lists-04.txt}

This left-inclusive behavior is kind of confusing, but there's nothing
I can really do about it. If you reference \code{x[i:j]} with
$i \ge j$, then you get an empty list.

You can extend the range syntax a little further, and insert a
``step'', meaning have it jump more than 1 spot at a time:

\begin{lstlisting}[language=Python]
>>> x = 'hello world'
>>> x[::]
'hello world'
>>> x[::2]
'hlowrd'
>>> x[::3]
'hlwl'
>>> x[::4]
'hor'
\end{lstlisting}

A step of $1$ works exactly as you'd expect

\begin{lstlisting}[language=Python]
>>> x[::1]
'hello world'
\end{lstlisting}

You can have the step be negative to work backward.

\begin{lstlisting}[language=Python]
>>> x[::-1]
'dlrow olleh'
>>> x[::-2]
'drwolh'
>>> x[::-3]
'dooe'
>>> 
\end{lstlisting}

(Hint: this is how we'll reverse the list later on).

To get the length of a list (or a string), use \code{len()}:

\lstin{lists-05.txt}

You can add two lists together using \code{+}:

\lstin{lists-06.txt}

You can (on the right) multiply a list by an integer to repeat the list:

\lstin{lists-07.txt}

Proper lists (i.e. not strings) do not need to contain elements of the
same type:

\lstin{lists-08.txt}

To test if a value is an element of the list, use \code{in}:

\lstin{lists-09.txt}


\subsection{Exercises}

You're now ready to do most of the level-1 CodingBat exercises.

\cbex{http://codingbat.com/prob/p189441}{not_string}
\cbex{http://codingbat.com/prob/p149524}{missing_char}
\cbex{http://codingbat.com/prob/p153599}{front_back}
\cbex{http://codingbat.com/prob/p193507}{string_times}
\cbex{http://codingbat.com/prob/p147920}{front3}
\cbex{http://codingbat.com/prob/p165097}{front_times}
\begin{exercise}
  Do all of List-1: \url{http://codingbat.com/python/List-1}.
\end{exercise}
\begin{exercise}
  Do all of String-1: \url{http://codingbat.com/python/String-1}.
\end{exercise}

\section{Loops}

Yay! We're on the last topic in the chapter! After this, the problem
we initially set out to solve will be easy.

There are two types of loops:

\begin{enumerate}
\item \code{while}-loops, and
\item \code{for}-loops.
\end{enumerate}

\code{while}-loops are simpler, and every \code{for}-loop can be
expressed as a \code{while}-loop.

While loops are of the format:

\lstpy{while-b.py}

where \code{b} is a Boolean. You can easily make an infinite loop
with:

\lstpy{while-true.py}

However, you should never do that. Typically, what you're doing is
repeating an action until \code{b} becomes \False, then you move
on. For instance:

\lstpy{while-i-less-than-10.py}

What do you think the output of that program will be? Try it and see!

That loop can be much more easily expressed as a
\code{for}-loop. Before I go much further, it's worth mentioning the
\code{range()} function:

\lstpy{for-i-in-range-10.py}

That program has the same output as the code with the
\code{while}-loop.

\subsection{Generating a list}

A common problem is to generate a list of things, using a loop. For
instance, to generate the even positive integers less than (or equal
to) $20$:

\lstpy{even-ints-lt-20.py}

Try running it, and toying with the program, to try to figure out why
it works the way it does.

Python has some syntax, borrowed from mathematics, that makes
generating lists a bit more pleasant. That's called a
\term{comprehension}. In mathematics, $\Ns$ is the set of positive
integers. So, we could write
$$\scomp{2x \in \Ns}{x \in \Ns \text{ and } x \le 10}$$ to mean ``the
set of all numbers in $\Ns$ of the form $2x$, where $x$ is in $\Ns$
and $x$ is less than or equal to $10$.'' In Python, this is

\begin{lstlisting}[language=Python]
>>> [2*x for x in range(1, 11)]
[2, 4, 6, 8, 10, 12, 14, 16, 18, 20]
\end{lstlisting}

Haskell also has list comprehensions, which have a little bit more
similarity to the mathematical notation:

\begin{lstlisting}[language=Haskell]
Prelude> [2*x | x <- [1..10]]
[2,4,6,8,10,12,14,16,18,20]
\end{lstlisting}

I'll go through one example, then give you some exercises

\begin{example}
  (From \url{http://codingbat.com/prob/p166170}).

  Given an array (list) of integers, return the number of 9's in the
  list.

  \begin{solution}
    There are several ways to do this, but here are the two I thought
    of:

    \begin{enumerate}
    \item Create an integer, call it $z$, set it equal to $0$. Loop
      through the list. For each element $x$ in the list, if $x = 9$,
      then increment $z$. Else, continue. The code would look
      something like this:

      \pyfile{code/codingbat/warmup2/array_count9.py}

      That approach works (see \cref{fig:array_count9-1}).

    \item However, the Haskellish way to do it, which is technically
      less efficient, but is much clearer, is this:

      \pyfile{code/codingbat/warmup2/array_count9-2.py}

      This also works (see \cref{fig:array_count9-2})
    \end{enumerate}
  \end{solution}
\end{example}

\begin{figure}[h]
  \centering
  \answergraph{images/array_count9-1.png}
  \caption{Approach 1 to \code{array_count9}.}
  \label{fig:array_count9-1}
\end{figure}

\begin{figure}[h]
  \centering
  \answergraph{images/array_count9-2.png}
  \caption{Approach 2 to \code{array_count9}.}
  \label{fig:array_count9-2}
\end{figure}


\subsection{Exercises}

\cbex{http://codingbat.com/prob/p118366}{string_splosion}
\cbex{http://codingbat.com/prob/p110166}{array_front9}
\cbex{http://codingbat.com/prob/p170842}{double_char}

\section{Extraneous information}

Here are some things that will be useful down the line, and that we're
going to use now, and in the future. They don't naturally fit in
anywhere in the book.

\begin{enumerate}
\item In the REPL, you can summon documentation via the \code{help()}
  function:

  \lstpy{help.txt}
\item You can also find that documentation online.
\item Python supports \code{printf}-style formatting. For instance, if
  I want to tell Python to print the integer variable \code{x} with at
  least 4 spaces, with leading zeroes:

  \lstin{printf-formatting.txt}

  There are way too many printf options for me to exhaustively
  document. However, you can read \code{man printf} to get an idea of
  what each of the control characters does. The syntax in Python is a
  bit different. For instance, 

  \lstin{printf-2.txt}

\item You're not actually supposed to use \code{printf}-style
  formatting in Python 3 anymore. I can't figure out the new system,
  and I don't program in Python enough to care. Nonetheless,
  \code{printf}-style formatting still works. More importantly, most
  languages have \code{printf}-style formatting, so it's a good skill
  to pick up.

\item The thing that looked like a list, but with parentheses instead
  of square brackets, is called a \term{tuple}. In Python, they behave
  like lists, but have some stupid properties that sometimes make them
  a pain to deal with.

\item In mathematics, there's a huge difference between tuples,
  vectors, and sets. Python sort of has a difference between (the
  respective Python equivalents of) all of these, but the distinction
  is less clear.
\item Strings can use double quotes: \code{"..."} or single quotes:
  \code{'...'}. It only matters if you have a string with either
  character inside of the string, in which case you can \code{escape}
  it, by putting a backslash before the character. For instance,

  \lstpy{single-double-quotes.py}

\item There are a variety of ``escape'' sequences, which you can use
  to put weird characters in a string. See \cref{tbl:escape-chars} for
  more information.

  \begin{table}
    \centering
    \begin{tabu}{r|l}
      \xtb{Python} & \xtb{Meaning} \\
      \tabucline \\
      \code{\\n} & New line (also called ``line feed'') \\
      \code{\\t} & Tabulator character \\
      \code{\\r} & Carriage return (go to the beginning of the line) \\
    \end{tabu}
    \caption{A non-exhaustive list of escape characters. These are the
      only ones you'll ever use, though.}
    \label{tbl:escape-chars}
  \end{table}
\end{enumerate}

Okay, I believe we're now ready to tackle the original problem:

\begin{quotation}
  Our goal at the end of this chapter is to write a program that does
  the following things, in a loop:

  \begin{enumerate}
  \item Read a piece of text from the user.
  \item Print out the text but reversed.
  \item If the user inputs `exit', then the program exits, but before
    we print out the reversed text.
  \end{enumerate}
\end{quotation}

Cool! This is super easy now, right? Here's my solution:

\pyfile{code/ch02goal.py}

Type it in yourself, and verify that it works as expected.