\documentclass[12pt,letterpaper,oneside]{memoir}

\usepackage{amsmath}
\usepackage{amssymb}
\usepackage{amsthm}
\usepackage[backend=bibtex,date=iso8601,urldate=iso8601]{biblatex}
\usepackage{color}
\usepackage{datetime}
\usepackage{graphicx}
\usepackage{listings}
\usepackage{mathtools}
\usepackage{tabu}
\usepackage{url}

\usepackage[scaled=0.85]{FiraMono}
\usepackage[T1]{fontenc}

\usepackage[hidelinks]{hyperref}
\usepackage{cleveref}

\definecolor{mygreen}{rgb}{0.3,0.6,0.3}
\definecolor{mygray}{rgb}{0.8,0.8,0.8}
\definecolor{mymauve}{rgb}{0.58,0,0.82}
\lstset{ %
  % backgroundcolor=\color{white},   % choose the background color; you must add \usepackage{color} or \usepackage{xcolor}
  basicstyle=\footnotesize\ttfamily,     % the size of the fonts that are used for the code
  breakatwhitespace=false,         % sets if automatic breaks should only happen at whitespace
  breaklines=true,                 % sets automatic line breaking
  captionpos=\null,                    % sets the caption-position to bottom
  commentstyle=\color{mygreen},    % comment style
  deletekeywords={...},            % if you want to delete keywords from the given language
  escapeinside={\%*}{*)},          % if you want to add LaTeX within your code
  extendedchars=true,              % lets you use non-ASCII characters; for 8-bits encodings only, does not work with UTF-8
  frame=single,                    % adds a frame around the code
  keepspaces=true,                 % keeps spaces in text, useful for keeping indentation of code (possibly needs columns=flexible)
  keywordstyle=\bfseries,       % keyword style
  % language=\null,                % the language of the code
  % morekeywords={*,...},            % if you want to add more keywords to the set
  numbers=left,                    % where to put the line-numbers; possible values are (none, left, right)
  numbersep=5pt,                   % how far the line-numbers are from the code
  numberstyle=\tiny\ttfamily,    % the style that is used for the line-numbers
  postbreak=\raisebox{0ex}[0ex][0ex]{\ensuremath{\color{red}\hookrightarrow\space}},
  rulecolor=\color{mygray},        % if not set, the frame-color may be changed on line-breaks within not-black text (e.g. comments (green here))
  showspaces=false,                % show spaces everywhere adding particular underscores; it overrides 'showstringspaces'
  showstringspaces=false,          % underline spaces within strings only
  showtabs=false,                  % show tabs within strings adding particular underscores
  stepnumber=1,                    % the step between two line-numbers. If it's 1, each line will be numbered
  stringstyle=\color{mymauve},     % string literal style
  tabsize=2,                       % sets default tabsize to 2 spaces
  title=\lstname,                   % show the filename of files included with \lstinputlisting; also try caption instead of title
  caption=\lstname ,                  % show the filename of files included with \lstinputlisting; also try caption instead of title
}

\addbibresource{pfs.bib}
\nocite{*}
\yyyymmdddate
\renewcommand*{\dateseparator}{-}

\theoremstyle{definition}
\newtheorem{remark}{Remark}[section]
\newtheorem{example}[remark]{Example}
\newtheorem{exercise}{Exercise}[section]

\theoremstyle{plain}
\newtheorem{axiom}[remark]{Axiom}
\newtheorem{definition}[remark]{Definition}
\newtheorem{lemma}[remark]{Lemma}
\newtheorem{corollary}[remark]{Corollary}
\newtheorem{theorem}[remark]{Theorem}
\newenvironment{solution}{\textbf{Solution. }}{\qed}

\newcommand{\answergraph}{\includegraphics[width=\textwidth]}

\newcommand{\code}[1]{\lstinline{#1}}
\newcommand{\lstin}[1]{\lstinputlisting{listings/#1}}
\newcommand{\lstpy}[1]{\lstinputlisting[language=Python]{listings/#1}}
\newcommand{\pyfile}[1]{\lstinputlisting[language=Python]{#1}}
\newcommand{\term}{\textbf}
\newcommand{\xtb}{\textbf}
\renewcommand{\epsilon}{\varepsilon}
% \renewcommand{\phi}{\varphi}
\newcommand{\True}{\code{True}}
\newcommand{\False}{\code{False}}
\newcommand{\Z}{\mathbb{Z}}
\newcommand{\N}{\mathbb{N}}
\newcommand{\Ns}{\mathbb{N^{+}}}
\newcommand{\abs}[1]{\left|#1\right|}
\newcommand{\norm}{\abs}
\newcommand{\braces}[1]{\left\{#1\right\}}
\newcommand{\parens}[1]{\left(#1\right)}
\newcommand{\brackets}[1]{\left[#1\right]}
\newcommand{\mset}[1]{\braces{\, #1 \,}}
\newcommand{\scomp}[2]{\mset{#1 \::\: #2}}
\newcommand{\ctext}[1]{$$\text{#1}$$}
\newcommand{\cbex}[2]{\begin{exercise}Do \url{#1}. (\code{#2})\end{exercise}}
\newcommand{\subof}{\subseteq}
\newcommand{\psubof}{\subsetneq}
\newcommand{\supof}{\supseteq}
\newcommand{\psupof}{\supsetneq}
\newenvironment{pylisting}{\begin{lstlisting}[language=Python]}{\end{lstlisting}}

\begin{document}
\title{Python from Scratch}
\author{Peter Harpending \texttt{<peter.harpending@utah.edu>}}
\maketitle

\noindent \hrulefill

Copyright \copyright\ 2016 Peter Harpending.

This work is licensed under the Creative Commons
Attribution-ShareAlike 4.0 International License. To view a copy of
this license, visit
\url{http://creativecommons.org/licenses/by-sa/4.0/}.

Source code for the book: \url{https://github.com/pharpend/pfs}.

Code examples in the book are licensed under the ISC License:

\begin{lstlisting}
Copyright (c) 2016, Peter Harpending <peter.harpending@utah.edu>

Permission to use, copy, modify, and/or distribute this software for
any purpose with or without fee is hereby granted, provided that the
above copyright notice and this permission notice appear in all
copies.

THE SOFTWARE IS PROVIDED "AS IS" AND THE AUTHOR DISCLAIMS ALL
WARRANTIES WITH REGARD TO THIS SOFTWARE INCLUDING ALL IMPLIED
WARRANTIES OF MERCHANTABILITY AND FITNESS. IN NO EVENT SHALL THE
AUTHOR BE LIABLE FOR ANY SPECIAL, DIRECT, INDIRECT, OR CONSEQUENTIAL
DAMAGES OR ANY DAMAGES WHATSOEVER RESULTING FROM LOSS OF USE, DATA OR
PROFITS, WHETHER IN AN ACTION OF CONTRACT, NEGLIGENCE OR OTHER
TORTIOUS ACTION, ARISING OUT OF OR IN CONNECTION WITH THE USE OR
PERFORMANCE OF THIS SOFTWARE.
\end{lstlisting}

\noindent \hrulefill

\tableofcontents
\chapter{Hi, there!}

This is a beginner-level Python ``book''. It's closer to a series of
rambling notes at this point. The intent of the book is to teach you
basic Python, as well as establish an appropriate mathematical
background.

Each chapter starts with a goal, and perhaps an outline of steps we'll
take to meet that goal. The chapters are designed so you can (ideally)
knock out one section in a reasonable sitting (an hour or two).

The target audience is people with no background in programming, and
minimal background in pure mathematics.

In this first chapter, we're going to get you set up, and then write
some very simple programs in Python.

\section{Getting set up}

You will want to install all of these things:

\begin{enumerate}
\item the Python 3 environment;
\item a plain-text editor;
\item if you're on Windows, Cygwin: \url{https://www.cygwin.com/}.
\end{enumerate}

If you're on a UNIX-like system, such as Linux, OS X, or a BSD system,
Python is probably already installed, and a text editor and terminal
emulator are almost certainly installed. The only hiccup might be the
version of Python.

On Windows, you'll want to download the latest Python 3 release from
here: \url{https://www.python.org/downloads/windows/}. (See
\cref{fig:python3-windows}.)

As for a ``plain-text editor'', that means a program to edit text
files that is critically \xtb{not a word processor}. There are
hundreds of text editors, and experienced programmers are usually
rather opinionated on their favorite text editor.

\begin{enumerate}
\item For the average new programmer, I'd recommend Atom:
  \url{https://atom.io/}. Atom is not a terrible choice for
  experienced developers, either.
\item If you are willing to learn how to use it, Vim is an excellent
  terminal-based editor: \url{http://www.vim.org/}. The
  poorly-designed website is not reflective of the quality of the text
  editor. Vim is probably the most popular text editor.
\item If you have several months to spare, and are already an
  experienced programmer, GNU Emacs is also very popular:
  \url{http://www.gnu.org/software/emacs/}. (This is what I use). In
  this case, the fast, and well-designed website is not reflective of
  the quality of the text editor.
\item If you are on Windows, I've heard good things about Notepad++:
  \url{https://notepad-plus-plus.org/}, although I've never tried it
  for more than an hour or two.
\end{enumerate}

\begin{figure}[ht]
  \centering
  \answergraph{images/python3-windows-gimped.png}
  \caption{Link you should click on for Python 3, iff you're using
    Windows.}
  \label{fig:python3-windows}
\end{figure}

Throughout this book, you'll need to run a lot of commands in the
terminal. On UNIX-like systems, you can open an application called
``Terminal,'' ``Terminal Emulator,'' or some variant thereof, and use
that. On Windows, use Cygwin. I will prefix commands you should run in
the terminal with a dollar sign. Expected output of the program
follows in subsequent lines.

\lstin{uname.txt}

\begin{remark}
  The red arrow just indicates that the previous line overflows onto
  the next line. It does not indicate an actual output character.
\end{remark}

\begin{remark}
  It's extremely important that you type in code examples and
  terminal commands yourself, and observe the results, rather than
  glancing at them in the book, or copying \& pasting.
\end{remark}

To test to see if Python 3 is installed on your system, first run
\code{which python} in a terminal.

\lstin{which-python.txt}

Then, run \code{python --version} to make sure it's using Python 3:

\lstin{python-version.txt}

On my system (Ubuntu 16.04), Python 2 is still the default. However,
the command \code{python3} gets me to Python 3.

\lstin{python3-version.txt}

The command to run Python 3 will vary from one system to another, so
therefore I will just say ``run the appropriate \code{python}
command,'' to refer to \code{python}, \code{python3}, or whatever it
is on your system.

If you run \code{python3} with no arguments, you'll get Python's
\term{Read/Eval/Print Loop}, or \term{REPL}.

\lstin{python3-repl.txt}

\begin{remark}
  The three `greater-than' signs are the standard Python REPL
  prompt. Therefore, if I want to indicate that you should type
  something into the Python REPL, I'll just prefix the line with
  \code{>>>}.
\end{remark}

At the REPL, you can type in a line of Python code. Python will
evaluate it and print the result. Then, under most circumstances,
Python will prompt you for another line of code.\footnote{The
  circumstance in which Python won't ask you for a new line of code is
  if you tell Python to quit via the \code{exit} command.} Examples:

\lstpy{hello-world-and-such.txt}

\begin{remark}
  The syntax highlighting is just there to help with readability; it
  doesn't actually mean anything.
\end{remark}

\subsection{Saving in files}

You can type entire programs into the REPL, and it will evaluate
them. However, you will usually want to save them in a file, so you
can use them later, and not have to type them out every time. Before
we do that, you need a small amount of background information.

I'd recommend creating a special directory for the programs in this
text. If you can't think of a name, I'd suggest \code{code}. To make a
directory, use the \code{mkdir} command:

\lstin{mkdir.txt}

There are a lot of UNIX commands, but here are the commands you'll
actually need to know as we move forward:

\begin{enumerate}
\item \code{mkdir} is for ``make directory.''
\item \code{cd} is for ``change directory.''
\item \code{pwd} is for ``print working directory.'' (Prints the
  current directory out).
\item \code{ls} is for ``list.'' (Prints the contents of the current
  directory)
\item \code{cp} is for ``copy.''
\item \code{mv} is for ``move.'' (Or ``rename,'' if you prefer.)
\item Most importantly, \code{man} is for ``manual.'' It will print
  out the manual for any command. I highly recommend glancing through
  the manual for all of the commands I just mentioned. For instance:

  \lstin{man-pwd.txt}

  To exit the manual, press \code{q}. The \code{#} indicates that the
  rest of the line is a \term{comment}, and is ignored by the shell
  (program that interprets \& executes terminal commands).

  One of the most important options for \code{man} is \code{man -k},
  which searches for manuals, or, in this case \term{man pages}.

  \lstin{man-k.txt}
\end{enumerate}

\paragraph{Your first Python program run from a file.}

Now that you know the basics of UNIX, we're going to run a very small
program from a file.

Python files typically have the \code{.py} file extension. Here's the
\code{hello world} program in a file:

\pyfile{code/hello_world.py}

Use your text editor to write (yes, write it yourself, don't copy \&
paste) that program to a file called \code{hello_world.py}, then run
this command:

\lstin{python3-hello_world.txt}

\begin{remark}
  The triple-quoted string at the beginning of the file is a
  \term{block comment}. It serves to explain what the file does. You
  should get in the habit of putting a block comment at the top of all
  of your Python files explaining what the file does. In fact, later
  on, when we are writing functions, we'll write a block comment in
  each function explaining what the function does.
\end{remark}

\subsection{Exercises}

\begin{exercise}
  Write \& run programs that print out the following strings:

  \begin{enumerate}
  \item \code{The quick brown fox jumped over the lazy dog.}
  \item \code{Programming is not actually all that hard.}
  \item \code{Nixon did nothing wrong.}
  \end{enumerate}
\end{exercise}

\begin{exercise}
  Calculate the result of each of the following expressions using
  Python. (Doing this in the REPL is fine):

  \begin{enumerate}
  \item $3 \times 2$
  \item $11 \times 18$
  \item $\frac{541}{241}$
  \item $100 + 847 - 17$
  \end{enumerate}
\end{exercise}

\section{Standard IO}

Our goal at the end of this section is to write a function that does
the following things, in a loop:

\begin{enumerate}
\item Read a piece of text from the user.
\item Print out the text but reversed.
\end{enumerate}

If the user inputs \code{exit}, then the program exits.

Vocabulary term: a piece of text is called a \term{string}.

To print out the text, use the \code{print()} function. To read text
input, we use the \code{input()} function. Examples:

\lstpy{input-example.txt}

Knowing about the function \code{input()} is a piece of trivia, that,
in the grand scheme of things, is unimportant. However, being able to
apply the thought process and techniques in this section to similar
problems is very important.

The broadest technique in programming is to break a problem down into
smaller problems. Here is how I would break down this problem:

\begin{enumerate}
\item start up the program;
\item read a string from the user using \code{input()};
\item if the string is \code{exit}, then die;
\item else, reverse the string;
\item print the reversed string;
\item go back to 2.
\end{enumerate}

There are two important concepts in my breakdown:

\begin{enumerate}
\item the \term{variable}, i.e. storing and changing a value;
\item the \term{conditional}, i.e. if A then B else C;
\item the \term{loop}, i.e. ``go back to 2.''
\end{enumerate}

These two concepts are examples of \term{control flow
  structures}. We'll learn more about them in this section.

\chapter{Standard IO}

Our goal at the end of this chapter is to write a program that does
the following things, in a loop:

\begin{enumerate}
\item Read a piece of text from the user.
\item Print out the text but reversed.
\item If the user inputs `exit', then the program exits, but before
  we print out the reversed text.
\end{enumerate}

Vocabulary term: a piece of text is called a \term{string}.

To print out the text, use the \code{print()} function. To read text
input, we use the \code{input()} function. Examples:

\lstpy{input-example.txt}

Knowing about the function \code{input()} is a piece of trivia, that,
in the grand scheme of things, is unimportant. However, being able to
apply the thought process and techniques in this section to similar
problems is very important.

The broadest technique in programming is to break a problem down into
smaller problems. Here is how I would break down this problem:

\begin{enumerate}
\item start up the program;
\item read a string from the user using \code{input()};
\item if the string is \code{exit}, then die;
\item else, reverse the string;
\item print the reversed string;
\item go back to 2.
\end{enumerate}

There are four important concepts in my breakdown:

\begin{enumerate}
\item the \term{variable}, i.e. storing and changing a value;
\item the \term{conditional}, i.e. if A then B else C;
\item the \term{loop}, i.e. ``go back to 2;''
\item a \term{subroutine}, where we delegate repetitive tasks to
  smaller programs stored inside our main program.
\end{enumerate}

The latter three concepts are examples of \term{flow control
  structures}. We'll learn more about them in this
chapter. Conditional statements are broad enough that they require
their own section.

\subsection{Variables}

The first thing is a variable. Most people are familiar with the
concept of a variable from mathematics courses. In most programming
languages, a variable can actually vary. So, for instance, in math,
this is (more or less) nonsense:

\begin{align*}
  a & = 2 \\
  a & = a + 10 \\
  a & \text{ is now equal to 12.}
\end{align*}

At the very least, it's rather uncommon. However, in most programming
languages, it's extremely common. It's the most fundamental and common
thing you do. Here's how we'd represent that in Python:

\lstpy{python-var-1.txt}

Every variable (indeed, every value) has a \term{type}. To get the
type of an object in Python, you use the \code{type()} function:

\lstpy{python-types-1.txt}

\begin{remark}
  In Python, types are less important, because Python uses \term{weak
    typing}, and \term{dynamic typing}, where Python does not actually
  determine the type of each object until it actually runs the
  code. However, in certain programming languages (languages with
  \term{static typing} and \term{strong typing}), types are extremely
  important.

  Weak typing is a fundamental weakness of Python, and indeed all
  weakly typed languages, including Ruby. Small mental errors can
  translate into impossible-to-find bugs in your code, that may take
  several days to find, let alone fix. This is why I strongly prefer
  static and strongly typed languages, such as Haskell. In Haskell,
  most bugs are caught before you even compile the software.

  If you take strong typing to the extreme, you get languages like
  Idris and Coq, where the type system is strong enough that you can
  actually mathematically prove the correctness of your program with
  the type system.
\end{remark}

\begin{remark}
  Fair warning: my experience is predominantly in \term{functional}
  programming languages, which do not have the concept of varying
  variables. Therefore, much of the code I write will not use the
  ``Pythonic'' method, and will instead employ a more Haskell-like
  approach.
\end{remark}

\begin{exercise}
  What do you suppose the types of these values are? Check your
  guesses in the REPL:

  \begin{enumerate}
  \item \code{2 + 2}
  \item \code{28 / 13}
  \item \code{28 / 14}
  \item \code{'hello, my name is joe'}
  \end{enumerate}
\end{exercise}

\begin{exercise}
  What do you think the following program does?

  \pyfile{code/print_plus_3.py}

  \xtb{Hint}: \emph{\code{int(x)} takes any value and tries to convert
    it to an integer. It might be worth toying around with
    \code{int()} in the REPL.}

  Try it and see.
\end{exercise}

\section{Conditionals and Boolean algebra}

Conditional statements, i.e. if/then/else, are based on
\term{Booleans}\footnote{``Boolean'' is capitalized because it is
  named after an English mathematician, George Boole.}.

\begin{definition}
  A Boolean is either true or false.
\end{definition}

The study and manipulation of Booleans is called \term{Boolean
  algebra}. Let's explore Booleans in Python:

\lstpy{python-bools-1.txt}

In our goal, we want to be able to create a Boolean that tells us
whether or not the string is `exit'.

\subsection{Combining Booleans}

Booleans are kind of useless unless you can

\begin{enumerate}
\item combine them to make new Booleans, and
\item use them to determine whether or not to do something.
\end{enumerate}

There are three common ways to combine Booleans:

\begin{enumerate}
\item negation, where \code{not True = False}, and \code{not False =
    True};
\item an ``and'' statement, such as \code{a and b}, which is true if
  and only if both \code{a} and \code{b} are true;
\item an ``or'' statement, such as \code{a or b}, which is true if and
  only if at least one of \code{a} and \code{b} is true.
\end{enumerate}

\lstpy{python-bools-1.txt}

A common exercise is to build a \term{truth table} for the
operations. I've already given you some of the answers with my example

\begin{table}
  \centering
  \begin{tabu}{r|c|c}
    \code{and} & \code{True} & \code{False} \\
    \tabucline \\
    \code{True} & \code{True} & \code{False} \\
    \tabucline \\
    \code{False} & & \\
  \end{tabu}
  \caption{Partially filled in truth table for \code{and}.}
  \label{tbl:truth-table-and}
\end{table}

\begin{table}
  \centering
  \begin{tabu}{r|c|c}
    \code{or} & \code{True} & \code{False} \\
    \tabucline \\
    \code{True} & & \code{True} \\
    \tabucline \\
    \code{False} & & \code{False} \\
  \end{tabu}
  \caption{Partially filled in truth table for \code{or}.}
  \label{tbl:truth-table-or}
\end{table}

The next thing is negation:

\lstpy{python-negation-1.txt}

This immediately gives rise to the following lemma, which is obvious:

\begin{lemma}
  \label{thm:double-neg-bool}
  For all Booleans \code{a}, \code{not (not a)} is equal to \code{a}.
\end{lemma}

It is not all that interesting. However, the following properties,
called \term{De Morgan's laws}, are:

\begin{lemma}
  \label{thm:de-morgan-1}
  For all Booleans \code{a} and \code{b}, \code{not (a and b)} is
  equal to \code{(not a) or (not b)}.
\end{lemma}

Think about it:

\begin{enumerate}
\item I told you that I went to the park and the grocery store;
\item but you know that I didn't go to the park, or that I didn't go
  to the grocery store;
\item then I lied to you.
\end{enumerate}

The converse is also true:

\begin{enumerate}
\item I told you that I didn't go to the park or I didn't go to the grocery store;
\item but you know that I did in fact go to the park, and that I did
  go to the grocery store;
\item then I lied to you.
\end{enumerate}

It might help to compute a few of these values in the REPL, or just
make a truth table (\cref{exc:de-morgan-table-1}).

This corollary can be obtained by combining \cref{thm:double-neg-bool}
and \cref{thm:de-morgan-1}:

\begin{corollary}
  \label{thm:de-morgan-2}
  For all Booleans \code{a} and \code{b}, \code{not (a or b)} is equal
  to \code{(not a) or (not b)}.
\end{corollary}

If you're not convinced, think about it for a second, or build a truth
table (\cref{exc:de-morgan-table-2}).

Next, we have the \term{distributive laws}:

\begin{lemma}
  For all Booleans \code{a}, \code{b}, and \code{c}, \code{a and (b or
    c)} is equal to \code{(a and b) or (a and c)}.
\end{lemma}

This is similar to the distributive law in math, for real
numbers: $$a(b + c) = ab + ac.$$ This is also true:

\begin{lemma}
  For all Booleans \code{a}, \code{b}, and \code{c}, \code{a or (b and
    c)} is equal to \code{(a or b) and (a or c)}.
\end{lemma}

The analog doesn't hold in math: $$a + bc \ne (a + b)(b + c).$$

\subsection{Testing for equality, and other mathematical tests}

It's tiring to put ``is equal to'' every time. Instead, Python has the
\code{==} operator:

\lstpy{python-equals-equals.txt}

\begin{remark}
  It's important to note that \code{a == b} \emph{tests} to see if
  \code{a} and \code{b} are equal, whereas \code{a = b} \emph{assigns}
  \code{a} to whatever \code{b} is.
\end{remark}

The analog of $\ne$ is \code{!=}:

\begin{lstlisting}[language=Python]
>>> 2 != 4
True
>>> 2 != 2
False
\end{lstlisting}

\Cref{tbl:math-python} might be helpful.

\begin{table}
  \centering
  \begin{tabu}{r|l}
    \textbf{Math} & \textbf{Python} \\
    \tabucline \\
    $a := b$ & \code{a = b} \\
    $a = b$ & \code{a == b} \\
    $a \ne b$ & \code{a != b} \\
    $a > b$ & \code{a > b} \\
    $a \ge b$ & \code{a >= b} \\
    $a < b$ & \code{a < b} \\
    $a \le b$ & \code{a <= b} \\
  \end{tabu}
  \caption{Mathematical statements, and their equivalent in Python}
  \label{tbl:math-python}
\end{table}

\subsection{Exercises}

\begin{exercise}
  Fill in \cref{tbl:truth-table-and}.
\end{exercise}
\begin{exercise}
  Fill in \cref{tbl:truth-table-or}.
\end{exercise}
\begin{exercise}
  \label{exc:and-comm}
  Do you think the following property is true?

  \emph{For all Booleans \code{a} and \code{b}, \code{a and b} equals
    \code{b and a}.}

  Check your truth table to see.
\end{exercise}
\begin{exercise}
  Do the same as \cref{exc:and-comm}, but for \code{or}.
\end{exercise}
\begin{exercise}
  \label{exc:and-assoc}
  Do you think the following property is true?

  \emph{For all Booleans \code{a}, \code{b}, and \code{c} \code{a and
      (b and c)} equals
    \code{(a and b) and c)}.}

  Constructing a truth table here would require 3 dimensions, so just
  convince yourself whether or not it's true, maybe trying things in
  the REPL.
\end{exercise}
\begin{exercise}
  Do the same as \cref{exc:and-assoc}, but for \code{or}.
\end{exercise}
\begin{exercise}
  \label{exc:de-morgan-table-1}
  Build two truth tables for \cref{thm:de-morgan-1}. That is, build
  one for \code{not (a and b)}, then build a separate truth table for
  \code{(not b) or (not a)}, and then verify that they are identical.

  \begin{center}
    \begin{tabu}{r|c|c}
      \code{not (a and b)} & \code{True} & \code{False} \\
      \tabucline \\
      \code{True} && \\
      \tabucline \\
      \code{False} && \\
    \end{tabu}
  \end{center}
  \begin{center}
    \centering
    \begin{tabu}{r|c|c}
      \code{(not a) or (not b)} & \code{True} & \code{False} \\
      \tabucline \\
      \code{True} && \\
      \tabucline \\
      \code{False} && \\
    \end{tabu}
  \end{center}
\end{exercise}
\begin{exercise}
  \label{exc:de-morgan-table-2}
  Build two truth tables for \cref{thm:de-morgan-2}, in similar fashion
  to \cref{exc:de-morgan-table-1}:

  \begin{center}
    \centering
    \begin{tabu}{r|c|c}
      \code{not (a or b)} & \code{True} & \code{False} \\
      \tabucline \\
      \code{True} && \\
      \tabucline \\
      \code{False} && \\
    \end{tabu}
  \end{center}

  \begin{center}
    \centering
    \begin{tabu}{r|c|c}
      \code{(not a) and (not b)} & \code{True} & \code{False} \\
      \tabucline \\
      \code{True} && \\
      \tabucline \\
      \code{False} && \\
    \end{tabu}
  \end{center}
\end{exercise}
\begin{exercise}
  Write a program that takes a number (via \code{input()}), and prints
  out whether or not it's greater than $34$.
\end{exercise}

\section{If/then/else}

Now that you understand what Booleans are, we need to be able to
instruct Python to

\begin{enumerate}
\item given a Boolean \code{b};
\item if \code{b} is true, do something;
\item if \code{b} is false, do something else.
\end{enumerate}

In Python, this is achieved via the \code{if} clause. The general
syntax is

\begin{lstlisting}[language=Python]
if b:
    # do something here
else:
    # do something different
\end{lstlisting}

The \code{#} represents a comment, which is ignored.

\begin{remark}
  The above is not valid code, because you'd need actual code inside
  the \code{if} and \code{else} blocks, rather than just a comment.
\end{remark}

\begin{remark}
  Please note the trailing colons after \code{if b} and
  \code{else}. It's really easy to forget them.
\end{remark}

Let's look at a simple program that takes in two numbers from the
user, and checks if their sum is $20$.

\pyfile{code/sum_is_twenty.py}

Here's an example input:

\lstin{sum_is_twenty.txt}

Often times, if the first condition fails to be true, you'll want to
test for other conditions inside the \code{else} block:

\begin{lstlisting}[language=Python]
if b1:
    # do something here
else:
    # more conditions
    if b2:
        # do something here
    else:
        # more conditions
      if b3:
          # do something here
      else:
          # ...
\end{lstlisting}

This situation is rather common, so instead we have \code{elif}:

\begin{lstlisting}[language=Python]
if b1:
    # do something here
elif b2:
    # do something here
elif b3:
    # do something here
# ...
else:
    # finally, if everything fails, here
\end{lstlisting}
\chapter{The greatest common divisor}

Our goal at the end of this chapter is to write a function that
computes the greatest common divisor of two positive integers. I'll
try to develop the appropriate math background, and the appropriate
Python skills so that you'll understand what's going on at the end of
the chapter.

\section{Sets}

Sets are kind of like lists. There are a couple of critical
differences:

\begin{enumerate}
\item in sets, the order in which elements appear does not matter;
\item in sets, the number of times an element appears does not matter.
\end{enumerate}

For small enough sets, you can list the elements in curly braces:

\begin{align*}
  \mset{1,2,3,4,5} & \text{ is a set;} \\
  \mset{3,2,5,1,4} & \text{ is the same set;} \\
  \mset{3,5,4,1,2,3,5} & \text{ is also the same set.}
\end{align*}

We can verify this in the REPL:

\lstpy{1to5-sets-lists.txt}

In Python, to say that $1$ is in the set $\mset{1,2,3,4,5}$, we write

\lstpy{1-in-1to5.txt}

In math, we write $$1 \in \mset{1,2,3,4,5}.$$

In math, we have the notions of \term{subsets} and
\term{supersets}. That is,

\begin{definition}
  $X$ is a subset of $Y$, written $X \subof Y$, if and only if
  \ctext{for all $x \in X$, $x \in Y$.}
\end{definition}

\begin{definition}
  $X$ is a proper subset of $Y$, written $X \psubof Y$, if and only if
  \ctext{for all $x \in X$, $x \in Y$;} and, there is some $y \in Y$
  such that $y \notin X$.
\end{definition}

The Python analogues for sets are \code{<=} and \code{<},
respectively.

\begin{remark}
  The symbol $\subset$ is more popular, but kind of ambiguous. It
  \emph{usually} refers to improper subsets. However, the symbols
  $\subof$ and $\psubof$ are unambiguous, so I'll use those.
\end{remark}

For instance, $$\mset{2,3,4} \subof \mset{1,2,3,4,5}.$$ In Python,
this is

\lstpy{subof.txt}

\begin{remark}
  Sets in math are actually much more closely related to \emph{types},
  rather than lists. Why this is the case is largely beyond the scope
  of this book. It is worth mentioning, though.
\end{remark}

\term{Supersets} are defined analogously:

\begin{definition}
  $X$ is a superset of $Y$, written $$X \supof Y,$$ if and only if
  $Y \subof X$.
\end{definition}

\begin{definition}
  $X$ is a proper superset of $Y$, written $$X \psupof Y,$$ if and
  only if $Y \psubof X$.
\end{definition}

Set equality is defined as two sets containing precisely the same
elements. In other words,

\begin{definition}
  Given two sets $X$ and $Y$, $X$ is equal to $Y$, written $$X = Y,$$
  if and only if \ctext{$X \subof Y$ and $X \supof Y$.}
\end{definition}

Much like lists, there is a set containing no elements, called the
\term{null set}. The symbol varies from book to book, but it's usually
one of $\emptyset$, $\varnothing$, $\phi$, or some variant
thereof. I'm going to use $\emptyset$, because it's the most standard.

\subsection{Combining sets}

There are three common ways to combine sets to create new sets:

\begin{enumerate}
\item the \term{union} of two sets, where you take elements that are
  in either set (or both);
\item the \term{intersection} of two sets, where you take elements
  that are in both sets;
\item the \term{difference} of two sets, where you take elements that
  are in one set, but not the other.
\end{enumerate}

\section{Mathematical background}

First of all, what are the ``positive integers''? \term{Integers} are
whole numbers: $$\Z = \mset{0, \pm 1, \pm 2, \pm 3, \dots}.$$ The
\term{natural numbers} are the positive integers:
$$\N = \mset{1,2,3,\dots}.$$

\begin{remark}
  Some people use $0$ as the first natural number. It doesn't matter a
  whole not. In our case, it's more convenient to have $1$ as the
  first natural number.
\end{remark}

\subsection{Sets}

Before we go much further, I'd like to introduce the concept of a
\term{set}. A set is a collection of objects, called \term{elements},
with no notion of order or multiplicity. Finite sets with few elements
are typically denoted with braces: \ctext{$\mset{1,2,3,4}$ is a set.}
The things distinguishing sets from \term{lists} or \term{vectors}
are:

\begin{enumerate}
\item in a set, the order in which elements appear is irrelevant;
\item in a set, the number of times that an element appears is
  irrelevant.
\end{enumerate}

Therefore, all of the following are the same set:

\begin{align*}
  & \mset{1,2,3,4} \\
  = & \mset{1,1,2,2,3,3,4,4} && \text{(Each element is duplicated).} \\
  = & \mset{1, 3, 3, 4, 1, 4, 2} && \text{(Different order, and some duplication).}
\end{align*}

Given a set $A$, and an element $x$, to say \ctext{$x$ is an element
  of $A$,} we write $$x \in A.$$ If \ctext{$x$ is not an element of
  $A$,} then we write $$x \notin A.$$

\paragraph{Set comprehension notation.}

This notation is common:

\begin{equation*}
  \scomp{\text{what each element looks like}}{\text{conditions about
      each element}}
\end{equation*}

In Python, we have list comprehensions, which are similar:

\lstpy{comprehensions.txt}

\subsection{Functions}

A \term{mathematical function} takes elements from one set and maps
them to elements in another set. For instance,

\begin{align*}
  f & : \N \to \N \\
  f(x) & = x + 3
\end{align*}

Python has things called ``functions'', but they are slightly
different. Here's how we would define that function in Python:

\pyfile{code/x_plus_3.py}

The difference comes down to a property called \term{referential
  transparency}:

\begin{axiom}
  \label{foo}
  If $f : A \to B$, and $x, y \in A$, then, \ctext{if $x = y$, then
    $f(x) = f(y)$.}
\end{axiom}

Functions in Python do not usually have this property. For instance,

\lstpy{random-randint.txt}

I called the same ``function'' with the same arguments, and got a
different result. That cannot happen in a mathematical function.

The following definitions are useful computations that we'll be using:

\begin{definition}
  The \term{Cartesian product} of two sets $A$ and $B$ is the set of
  ordered pairs where the first value is in $A$, and the second value
  is in $B$. That is,
  $$A \times B = \scomp{(a, b)}{a \in A, \text{ and } b \in B}.$$
\end{definition}

\begin{definition}
    
\end{definition}

\subsection{Division with remainder}

This is the way you learned to divide in elementary school. We want a
mathematical function, that, given two positive integers $a, b \in
\N$, returns the \term{quotient} of $a$ and $b$, plus a
\term{remainder}, which might be $0$.

\begin{lemma}
  Given two positive integers $a, b \in \N$, with $a \ge b$ we can
  always write $$a = qb + r,$$ where $q \in \N$ is the
  \term{quotient}, and $r$ is the \term{remainder}. The remainder can
  be zero, but it must be less than $b$. Therefore,
  $$r \in \scomp{x \in \Z}{0 \le x < b}.$$
\end{lemma}


See \cref{foo}.

\printbibliography
\end{document}
