\documentclass[12pt,letterpaper,oneside]{memoir}

\usepackage{amsmath}
\usepackage{amssymb}
\usepackage{amsthm}
\usepackage[backend=bibtex,date=iso8601,urldate=iso8601]{biblatex}
\usepackage{color}
\usepackage{datetime}
\usepackage{graphicx}
\usepackage{listings}
\usepackage{mathtools}
\usepackage{url}

\usepackage[scaled=0.85]{FiraMono}
\usepackage[T1]{fontenc}

\usepackage[hidelinks]{hyperref}
\usepackage{cleveref}

\definecolor{mygreen}{rgb}{0.3,0.6,0.3}
\definecolor{mygray}{rgb}{0.8,0.8,0.8}
\definecolor{mymauve}{rgb}{0.58,0,0.82}
\lstset{ %
  % backgroundcolor=\color{white},   % choose the background color; you must add \usepackage{color} or \usepackage{xcolor}
  basicstyle=\footnotesize\ttfamily,     % the size of the fonts that are used for the code
  breakatwhitespace=false,         % sets if automatic breaks should only happen at whitespace
  breaklines=true,                 % sets automatic line breaking
  captionpos=\null,                    % sets the caption-position to bottom
  commentstyle=\color{mygreen},    % comment style
  deletekeywords={...},            % if you want to delete keywords from the given language
  escapeinside={\%*}{*)},          % if you want to add LaTeX within your code
  extendedchars=true,              % lets you use non-ASCII characters; for 8-bits encodings only, does not work with UTF-8
  frame=single,                    % adds a frame around the code
  keepspaces=true,                 % keeps spaces in text, useful for keeping indentation of code (possibly needs columns=flexible)
  keywordstyle=\bfseries,       % keyword style
  % language=\null,                % the language of the code
  % morekeywords={*,...},            % if you want to add more keywords to the set
  numbers=left,                    % where to put the line-numbers; possible values are (none, left, right)
  numbersep=5pt,                   % how far the line-numbers are from the code
  numberstyle=\tiny\ttfamily,    % the style that is used for the line-numbers
  postbreak=\raisebox{0ex}[0ex][0ex]{\ensuremath{\color{red}\hookrightarrow\space}},
  rulecolor=\color{mygray},        % if not set, the frame-color may be changed on line-breaks within not-black text (e.g. comments (green here))
  showspaces=false,                % show spaces everywhere adding particular underscores; it overrides 'showstringspaces'
  showstringspaces=false,          % underline spaces within strings only
  showtabs=false,                  % show tabs within strings adding particular underscores
  stepnumber=1,                    % the step between two line-numbers. If it's 1, each line will be numbered
  stringstyle=\color{mymauve},     % string literal style
  tabsize=2,                       % sets default tabsize to 2 spaces
  title=\lstname,                   % show the filename of files included with \lstinputlisting; also try caption instead of title
  caption=\lstname ,                  % show the filename of files included with \lstinputlisting; also try caption instead of title
}

\addbibresource{pfn.bib}
\nocite{*}
\yyyymmdddate
\renewcommand*{\dateseparator}{-}

\theoremstyle{definition}
\newtheorem{remark}{Remark}[section]
\newtheorem{definition}[remark]{Definition}
\newtheorem{exercise}{Exercise}[section]

\theoremstyle{plain}
\newtheorem{axiom}[remark]{Axiom}
\newtheorem{lemma}[remark]{Lemma}

\newcommand{\code}[1]{\lstinline{#1}}
\newcommand{\lstin}[1]{\lstinputlisting{listings/#1}}
\newcommand{\lstpy}[1]{\lstinputlisting[language=Python]{listings/#1}}
\newcommand{\pyfile}[1]{\lstinputlisting[language=Python]{#1}}
\newcommand{\answergraph}[1]{\includegraphics[width=0.8\textwidth]{#1}}
\newcommand{\term}{\textbf}
\newcommand{\xtb}{\textbf}

\newcommand{\Z}{\mathbb{Z}}
\newcommand{\N}{\mathbb{N}}
\newcommand{\braces}[1]{\left\{#1\right\}}
\newcommand{\parens}[1]{\left(#1\right)}
\newcommand{\brackets}[1]{\left[#1\right]}
\newcommand{\mset}[1]{\braces{\, #1 \,}}
\newcommand{\scomp}[2]{\mset{#1 \::\: #2}}
\newcommand{\ctext}[1]{$$\text{#1}$$}

\begin{document}
\title{Python from Scratch}
\author{Peter Harpending \texttt{<peter.harpending@utah.edu>}}
\maketitle

\noindent \hrulefill

Copyright \copyright\ 2016 Peter Harpending.

This work is licensed under the Creative Commons
Attribution-ShareAlike 4.0 International License. To view a copy of
this license, visit
\url{http://creativecommons.org/licenses/by-sa/4.0/}.

Code examples in the book are licensed under the ISC License:

\begin{lstlisting}
Copyright (c) 2016, Peter Harpending <peter.harpending@utah.edu>

Permission to use, copy, modify, and/or distribute this software for
any purpose with or without fee is hereby granted, provided that the
above copyright notice and this permission notice appear in all
copies.

THE SOFTWARE IS PROVIDED "AS IS" AND THE AUTHOR DISCLAIMS ALL
WARRANTIES WITH REGARD TO THIS SOFTWARE INCLUDING ALL IMPLIED
WARRANTIES OF MERCHANTABILITY AND FITNESS. IN NO EVENT SHALL THE
AUTHOR BE LIABLE FOR ANY SPECIAL, DIRECT, INDIRECT, OR CONSEQUENTIAL
DAMAGES OR ANY DAMAGES WHATSOEVER RESULTING FROM LOSS OF USE, DATA OR
PROFITS, WHETHER IN AN ACTION OF CONTRACT, NEGLIGENCE OR OTHER
TORTIOUS ACTION, ARISING OUT OF OR IN CONNECTION WITH THE USE OR
PERFORMANCE OF THIS SOFTWARE.
\end{lstlisting}

\noindent \hrulefill

\tableofcontents
\chapter{Hi, there!}

This is a beginner-level Python ``book''. It's closer to a series of
rambling notes at this point. The intent of the book is to teach you
basic Python, as well as establish an appropriate mathematical
background.

Each chapter starts with a goal, and perhaps an outline of steps we'll
take to meet that goal. The chapters are designed so you can (ideally)
knock out one section in a reasonable sitting (an hour or two).

The target audience is people with no background in programming, and
minimal background in pure mathematics.

In this first chapter, we're going to get you set up, and then write
some very simple programs in Python.

\section{Getting set up}

You will want to install all of these things:

\begin{enumerate}
\item the Python 3 environment;
\item a plain-text editor;
\item if you're on Windows, Cygwin: \url{https://www.cygwin.com/}.
\end{enumerate}

If you're on a UNIX-like system, such as Linux, OS X, or a BSD system,
Python is probably already installed, and a text editor and terminal
emulator are almost certainly installed. The only hiccup might be the
version of Python.

On Windows, you'll want to download the latest Python 3 release from
here: \url{https://www.python.org/downloads/windows/}. (See
\cref{fig:python3-windows}.)

As for a ``plain-text editor'', that means a program to edit text
files that is critically \xtb{not a word processor}. There are
hundreds of text editors, and experienced programmers are usually
rather opinionated on their favorite text editor.

\begin{enumerate}
\item For the average new programmer, I'd recommend Atom:
  \url{https://atom.io/}. Atom is not a terrible choice for
  experienced developers, either.
\item If you are willing to learn how to use it, Vim is an excellent
  terminal-based editor: \url{http://www.vim.org/}. The
  poorly-designed website is not reflective of the quality of the text
  editor. Vim is probably the most popular text editor.
\item If you have several months to spare, and are already an
  experienced programmer, GNU Emacs is also very popular:
  \url{http://www.gnu.org/software/emacs/}. (This is what I use). In
  this case, the fast, and well-designed website is not reflective of
  the quality of the text editor.
\item If you are on Windows, I've heard good things about Notepad++:
  \url{https://notepad-plus-plus.org/}, although I've never tried it
  for more than an hour or two.
\end{enumerate}

\begin{figure}[ht]
  \centering
  \answergraph{images/python3-windows-gimped.png}
  \caption{Link you should click on for Python 3, iff you're using
    Windows.}
  \label{fig:python3-windows}
\end{figure}

Throughout this book, you'll need to run a lot of commands in the
terminal. On UNIX-like systems, you can open an application called
``Terminal,'' ``Terminal Emulator,'' or some variant thereof, and use
that. On Windows, use Cygwin. I will prefix commands you should run in
the terminal with a dollar sign. Expected output of the program
follows in subsequent lines.

\lstin{uname.txt}

\begin{remark}
  The red arrow just indicates that the previous line overflows onto
  the next line. It does not indicate an actual output character.
\end{remark}

\begin{remark}
  It's extremely important that you type in code examples and
  terminal commands yourself, and observe the results, rather than
  glancing at them in the book, or copying \& pasting.
\end{remark}

To test to see if Python 3 is installed on your system, first run
\code{which python} in a terminal.

\lstin{which-python.txt}

Then, run \code{python --version} to make sure it's using Python 3:

\lstin{python-version.txt}

On my system (Ubuntu 16.04), Python 2 is still the default. However,
the command \code{python3} gets me to Python 3.

\lstin{python3-version.txt}

The command to run Python 3 will vary from one system to another, so
therefore I will just say ``run the appropriate \code{python}
command,'' to refer to \code{python}, \code{python3}, or whatever it
is on your system.

If you run \code{python3} with no arguments, you'll get Python's
\term{Read/Eval/Print Loop}, or \term{REPL}.

\lstin{python3-repl.txt}

\begin{remark}
  The three `greater-than' signs are the standard Python REPL
  prompt. Therefore, if I want to indicate that you should type
  something into the Python REPL, I'll just prefix the line with
  \code{>>>}.
\end{remark}

At the REPL, you can type in a line of Python code. Python will
evaluate it and print the result. Then, under most circumstances,
Python will prompt you for another line of code.\footnote{The
  circumstance in which Python won't ask you for a new line of code is
  if you tell Python to quit via the \code{exit} command.} Examples:

\lstpy{hello-world-and-such.txt}

\begin{remark}
  The syntax highlighting is just there to help with readability; it
  doesn't actually mean anything.
\end{remark}

\subsection{Saving in files}

You can type entire programs into the REPL, and it will evaluate
them. However, you will usually want to save them in a file, so you
can use them later, and not have to type them out every time. Before
we do that, you need a small amount of background information.

I'd recommend creating a special directory for the programs in this
text. If you can't think of a name, I'd suggest \code{code}. To make a
directory, use the \code{mkdir} command:

\lstin{mkdir.txt}

There are a lot of UNIX commands, but here are the commands you'll
actually need to know as we move forward:

\begin{enumerate}
\item \code{mkdir} is for ``make directory.''
\item \code{cd} is for ``change directory.''
\item \code{pwd} is for ``print working directory.'' (Prints the
  current directory out).
\item \code{ls} is for ``list.'' (Prints the contents of the current
  directory)
\item \code{cp} is for ``copy.''
\item \code{mv} is for ``move.'' (Or ``rename,'' if you prefer.)
\item Most importantly, \code{man} is for ``manual.'' It will print
  out the manual for any command. I highly recommend glancing through
  the manual for all of the commands I just mentioned. For instance:

  \lstin{man-pwd.txt}

  To exit the manual, press \code{q}. The \code{#} indicates that the
  rest of the line is a \term{comment}, and is ignored by the shell
  (program that interprets \& executes terminal commands).

  One of the most important options for \code{man} is \code{man -k},
  which searches for manuals, or, in this case \term{man pages}.

  \lstin{man-k.txt}
\end{enumerate}

\paragraph{Your first Python program run from a file.}

Now that you know the basics of UNIX, we're going to run a very small
program from a file.

Python files typically have the \code{.py} file extension. Here's the
\code{hello world} program in a file:

\pyfile{code/hello_world.py}

Use your text editor to write (yes, write it yourself, don't copy \&
paste) that program to a file called \code{hello_world.py}, then run
this command:

\lstin{python3-hello_world.txt}

\begin{remark}
  The triple-quoted string at the beginning of the file is a
  \term{block comment}. It serves to explain what the file does. You
  should get in the habit of putting a block comment at the top of all
  of your Python files explaining what the file does. In fact, later
  on, when we are writing functions, we'll write a block comment in
  each function explaining what the function does.
\end{remark}

\subsection{Exercises}

\begin{exercise}
  Write \& run programs that print out the following strings:

  \begin{enumerate}
  \item \code{The quick brown fox jumped over the lazy dog.}
  \item \code{Programming is not actually all that hard.}
  \item \code{Nixon did nothing wrong.}
  \end{enumerate}
\end{exercise}

\begin{exercise}
  Calculate the result of each of the following expressions using
  Python. (Doing this in the REPL is fine):

  \begin{enumerate}
  \item $3 \times 2$
  \item $11 \times 18$
  \item $\frac{541}{241}$
  \item $100 + 847 - 17$
  \end{enumerate}
\end{exercise}

\section{Standard IO}

Our goal at the end of this section is to write a function that does
the following things, in a loop:

\begin{enumerate}
\item Read a piece of text from the user.
\item Print out the text but reversed.
\end{enumerate}

If the user inputs \code{exit}, then the program exits.

Vocabulary term: a piece of text is called a \term{string}.

To print out the text, use the \code{print()} function. To read text
input, we use the \code{input()} function. Examples:

\lstpy{input-example.txt}

Knowing about the function \code{input()} is a piece of trivia, that,
in the grand scheme of things, is unimportant. However, being able to
apply the thought process and techniques in this section to similar
problems is very important.

The broadest technique in programming is to break a problem down into
smaller problems. Here is how I would break down this problem:

\begin{enumerate}
\item start up the program;
\item read a string from the user using \code{input()};
\item if the string is \code{exit}, then die;
\item else, reverse the string;
\item print the reversed string;
\item go back to 2.
\end{enumerate}

There are two important concepts in my breakdown:

\begin{enumerate}
\item the \term{variable}, i.e. storing and changing a value;
\item the \term{conditional}, i.e. if A then B else C;
\item the \term{loop}, i.e. ``go back to 2.''
\end{enumerate}

These two concepts are examples of \term{control flow
  structures}. We'll learn more about them in this section.

\chapter{Standard IO}

Our goal at the end of this chapter is to write a program that does
the following things, in a loop:

\begin{enumerate}
\item Read a piece of text from the user.
\item Print out the text but reversed.
\item If the user inputs `exit', then the program exits, but before
  we print out the reversed text.
\end{enumerate}

Vocabulary term: a piece of text is called a \term{string}.

To print out the text, use the \code{print()} function. To read text
input, we use the \code{input()} function. Examples:

\lstpy{input-example.txt}

Knowing about the function \code{input()} is a piece of trivia, that,
in the grand scheme of things, is unimportant. However, being able to
apply the thought process and techniques in this section to similar
problems is very important.

The broadest technique in programming is to break a problem down into
smaller problems. Here is how I would break down this problem:

\begin{enumerate}
\item start up the program;
\item read a string from the user using \code{input()};
\item if the string is \code{exit}, then die;
\item else, reverse the string;
\item print the reversed string;
\item go back to 2.
\end{enumerate}

There are four important concepts in my breakdown:

\begin{enumerate}
\item the \term{variable}, i.e. storing and changing a value;
\item the \term{conditional}, i.e. if A then B else C;
\item the \term{loop}, i.e. ``go back to 2;''
\item a \term{subroutine}, where we delegate repetitive tasks to
  smaller programs stored inside our main program.
\end{enumerate}

The latter three concepts are examples of \term{flow control
  structures}. We'll learn more about them in this
chapter. Conditional statements are broad enough that they require
their own section.

\section{Variables}

The first thing is a variable. Most people are familiar with the
concept of a variable from mathematics courses. In most programming
languages, a variable can actually vary. So, for instance, in math,
this is (more or less) nonsense:

\begin{align*}
  a & = 2 \\
  a & = a + 10 \\
  a & \text{ is now equal to 12.}
\end{align*}

At the very least, it's rather uncommon. However, in most programming
languages, it's extremely common. It's the most fundamental and common
thing you do. Here's how we'd represent that in Python:

\lstpy{python-var-1.txt}

Every variable (indeed, every value) has a \term{type}. To get the
type of an object in Python, you use the \code{type()} function:

\lstpy{python-types-1.txt}

\begin{remark}
  In Python, types are less important, because Python uses \term{weak
    typing}, and \term{dynamic typing}, where Python does not actually
  determine the type of each object until it actually runs the
  code. However, in certain programming languages (languages with
  \term{static typing} and \term{strong typing}), types are extremely
  important.

  Weak typing is a fundamental weakness of Python, and indeed all
  weakly typed languages, including Ruby. Small mental errors can
  translate into impossible-to-find bugs in your code, that may take
  several days to find, let alone fix. This is why I strongly prefer
  static and strongly typed languages, such as Haskell. In Haskell,
  most bugs are caught before you even compile the software.

  If you take strong typing to the extreme, you get languages like
  Idris and Coq, where the type system is strong enough that you can
  actually mathematically prove the correctness of your program with
  the type system.
\end{remark}

\begin{remark}
  Fair warning: my experience is predominantly in \term{functional}
  programming languages, which do not have the concept of varying
  variables. Therefore, much of the code I write will not use the
  ``Pythonic'' method, and will instead employ a more Haskell-like
  approach.
\end{remark}

\subsection{Legal variable names}

Python, and indeed every language, has rules about legal variable
names:

\begin{enumerate}
\item variable names \emph{must} start with a letter or an underscore;
\item variable names \emph{should} start with a lowercase letter or an
  underscore;
\item variable names may only contain letters, numbers, and underscores;
\item variable names are case-sensitive;
\item separate words in variables should use the \code{snake_case}
  convention (as opposed to the \code{idontknowwhatwordsare} or
  \code{camelCase} conventions).
\end{enumerate}

\begin{exercise}
  What do you suppose the types of these values are? Check your
  guesses in the REPL:

  \begin{enumerate}
  \item \code{2 + 2}
  \item \code{28 / 13}
  \item \code{28 / 14}
  \item \code{'hello, my name is joe'}
  \end{enumerate}
\end{exercise}

\begin{exercise}
  What do you think the following program does?

  \pyfile{code/print_plus_3.py}

  \xtb{Hint}: \emph{\code{int(x)} takes any value and tries to convert
    it to an integer. It might be worth toying around with
    \code{int()} in the REPL.}

  Try it and see.
\end{exercise}

\begin{exercise}
  Which of the following are legal variable names?

  \begin{enumerate}
  \item \code{Hello_my_name_is_Peter!}
  \item \code{Hello_my_name_is_Peter}
  \item \code{one+one_equals_2}
  \item \code{one_plus_one_equals_2}
  \item \code{n00b}
  \item \code{0perator}
  \end{enumerate}

  Try these out in the REPL (assign each name the value of \code{0} or
  something).
\end{exercise}

\section{Conditionals and Boolean algebra}

Conditional statements, i.e. if/then/else, are based on
\term{Booleans}\footnote{``Boolean'' is capitalized because it is
  named after an English mathematician, George Boole.}.

\begin{definition}
  A Boolean is either true or false.
\end{definition}

The study and manipulation of Booleans is called \term{Boolean
  algebra}. Let's explore Booleans in Python:

\lstpy{python-bools-1.txt}

In our goal, we want to be able to create a Boolean that tells us
whether or not the string is `exit'.

\subsection{Combining Booleans}

Booleans are kind of useless unless you can

\begin{enumerate}
\item combine them to make new Booleans, and
\item use them to determine whether or not to do something.
\end{enumerate}

There are three common ways to combine Booleans:

\begin{enumerate}
\item negation, where \code{not True = False}, and \code{not False =
    True};
\item an ``and'' statement, such as \code{a and b}, which is true if
  and only if both \code{a} and \code{b} are true;
\item an ``or'' statement, such as \code{a or b}, which is true if and
  only if at least one of \code{a} and \code{b} is true.
\end{enumerate}

\lstpy{python-bools-1.txt}

A common exercise is to build a \term{truth table} for the
operations. I've already given you some of the answers with my example

\begin{table}
  \centering
  \begin{tabu}{r|c|c}
    \code{and} & \code{True} & \code{False} \\
    \tabucline \\
    \code{True} & \code{True} & \code{False} \\
    \tabucline \\
    \code{False} & & \\
  \end{tabu}
  \caption{Partially filled in truth table for \code{and}.}
  \label{tbl:truth-table-and}
\end{table}

\begin{table}
  \centering
  \begin{tabu}{r|c|c}
    \code{or} & \code{True} & \code{False} \\
    \tabucline \\
    \code{True} & & \code{True} \\
    \tabucline \\
    \code{False} & & \code{False} \\
  \end{tabu}
  \caption{Partially filled in truth table for \code{or}.}
  \label{tbl:truth-table-or}
\end{table}

The next thing is negation:

\lstpy{python-negation-1.txt}

This immediately gives rise to the following lemma, which is obvious:

\begin{lemma}
  \label{thm:double-neg-bool}
  For all Booleans \code{a}, \code{not (not a)} is equal to \code{a}.
\end{lemma}

It is not all that interesting. However, the following properties,
called \term{De Morgan's laws}, are:

\begin{lemma}
  \label{thm:de-morgan-1}
  For all Booleans \code{a} and \code{b}, \code{not (a and b)} is
  equal to \code{(not a) or (not b)}.
\end{lemma}

Think about it:

\begin{enumerate}
\item I told you that I went to the park and the grocery store;
\item but you know that I didn't go to the park, or that I didn't go
  to the grocery store;
\item then I lied to you.
\end{enumerate}

The converse is also true:

\begin{enumerate}
\item I told you that I didn't go to the park or I didn't go to the grocery store;
\item but you know that I did in fact go to the park, and that I did
  go to the grocery store;
\item then I lied to you.
\end{enumerate}

It might help to compute a few of these values in the REPL, or just
make a truth table (\cref{exc:de-morgan-table-1}).

This corollary can be obtained by combining \cref{thm:double-neg-bool}
and \cref{thm:de-morgan-1}:

\begin{corollary}
  \label{thm:de-morgan-2}
  For all Booleans \code{a} and \code{b}, \code{not (a or b)} is equal
  to \code{(not a) or (not b)}.
\end{corollary}

If you're not convinced, think about it for a second, or build a truth
table (\cref{exc:de-morgan-table-2}).

Next, we have the \term{distributive laws}:

\begin{lemma}
  For all Booleans \code{a}, \code{b}, and \code{c}, \code{a and (b or
    c)} is equal to \code{(a and b) or (a and c)}.
\end{lemma}

This is similar to the distributive law in math, for real
numbers: $$a(b + c) = ab + ac.$$ This is also true:

\begin{lemma}
  For all Booleans \code{a}, \code{b}, and \code{c}, \code{a or (b and
    c)} is equal to \code{(a or b) and (a or c)}.
\end{lemma}

The analog doesn't hold in math: $$a + bc \ne (a + b)(b + c).$$

\subsection{CodingBat}

I mentioned at the beginning of the book that many of the exercises
are just to complete CodingBat exercises. This is because CodingBat
will check your answers for you, and give you immediate feedback. You
have \emph{almost} enough information to do the first one. There's one
critical piece of information missing, and that's the notion of a
\term{function}.

The syntax for functions is a bit difficult to explain. It will be
easier if I just write a few functions, then let you figure things
out.  To illustrate, I'll do the first couple of CodingBat exercises,
then leave others in the exercises.

\begin{example}
  (This is from \url{http://codingbat.com/prob/p173401}).

  We are given two variables: \code{weekday} and \code{vacation}, both
  of which are Booleans. We want to write a function that determines
  whether or not we sleep in. If it is the weekend (i.e. \code{not
    weekday}), or we are on vacation, then we sleep in. Else, we don't

  \begin{solution}
    CodingBat gives you the beginning of the function:

\begin{lstlisting}[language=Python]
def sleep_in(weekday, vacation):
    # your code goes here
\end{lstlisting}
    You need to write something in the body that, at the end, returns
    a Boolean determining whether or not we sleep in.

    We combine these two conditions with \code{or}

    \begin{enumerate}
    \item it's the weekend (i.e \code{not weekday});
    \item we're on vacation (i.e. \code{vacation}).
    \end{enumerate}

    \pyfile{code/codingbat/warmup1/sleep_in.py}

    This gives the correct result (see \cref{fig:sleep_in}).
  \end{solution}
\end{example}

\begin{figure}[h]
  \centering
  \answergraph{images/sleep_in.png}
  \caption{The result of our \code{sleep_in()} solution.}
  \label{fig:sleep_in}
\end{figure}

\begin{example}
  (From \url{http://codingbat.com/prob/p120546}).

  \begin{quote}
    We have two monkeys, a and b, and the parameters \code{a_smile}
    and \code{b_smile} indicate if each is smiling. We are in trouble
    if they are both smiling or if neither of them is smiling. Return
    \code{True} if we are in trouble.
  \end{quote}

  \begin{solution}
    There are two things, which can be combined with \code{or}:

    \begin{enumerate}
    \item both monkeys smiling (\code{a_smile and b_smile});
    \item neither monkey smiling (\code{(not a_smile) and (not
        b_smile)}).
    \end{enumerate}

    Therefore, our result is

    \pyfile{code/codingbat/warmup1/monkey_trouble.py}

    As you can see in \cref{fig:monkey_trouble}, this works!
  \end{solution}
\end{example}

\begin{figure}[h]
  \centering
  \answergraph{images/monkey_trouble.png}
  \caption{The result of running our \code{monkey_trouble()} solution
    in CodingBat.}
  \label{fig:monkey_trouble}
\end{figure}

There's more to cover, but this section is too long as it is, so here
are some exercises:

\subsection{Exercises}

\begin{exercise}
  Fill in \cref{tbl:truth-table-and}.
\end{exercise}
\begin{exercise}
  Fill in \cref{tbl:truth-table-or}.
\end{exercise}
\begin{exercise}
  \label{exc:and-comm}
  Do you think the following property is true?

  \emph{For all Booleans \code{a} and \code{b}, \code{a and b} equals
    \code{b and a}.}

  Check your truth table to see.
\end{exercise}
\begin{exercise}
  Do the same as \cref{exc:and-comm}, but for \code{or}.
\end{exercise}
\begin{exercise}
  \label{exc:and-assoc}
  Do you think the following property is true?

  \emph{For all Booleans \code{a}, \code{b}, and \code{c} \code{a and
      (b and c)} equals
    \code{(a and b) and c)}.}

  Constructing a truth table here would require 3 dimensions, so just
  convince yourself whether or not it's true, maybe trying things in
  the REPL.
\end{exercise}
\begin{exercise}
  Do the same as \cref{exc:and-assoc}, but for \code{or}.
\end{exercise}
\begin{exercise}
  \label{exc:de-morgan-table-1}
  Build two truth tables for \cref{thm:de-morgan-1}. That is, build
  one for \code{not (a and b)}, then build a separate truth table for
  \code{(not b) or (not a)}, and then verify that they are identical.

  \begin{center}
    \begin{tabu}{r|c|c}
      \code{not (a and b)} & \code{True} & \code{False} \\
      \tabucline \\
      \code{True} && \\
      \tabucline \\
      \code{False} && \\
    \end{tabu}
  \end{center}
  \begin{center}
    \centering
    \begin{tabu}{r|c|c}
      \code{(not a) or (not b)} & \code{True} & \code{False} \\
      \tabucline \\
      \code{True} && \\
      \tabucline \\
      \code{False} && \\
    \end{tabu}
  \end{center}
\end{exercise}
\begin{exercise}
  \label{exc:de-morgan-table-2}
  Build two truth tables for \cref{thm:de-morgan-2}, in similar fashion
  to \cref{exc:de-morgan-table-1}:

  \begin{center}
    \centering
    \begin{tabu}{r|c|c}
      \code{not (a or b)} & \code{True} & \code{False} \\
      \tabucline \\
      \code{True} && \\
      \tabucline \\
      \code{False} && \\
    \end{tabu}
  \end{center}

  \begin{center}
    \centering
    \begin{tabu}{r|c|c}
      \code{(not a) and (not b)} & \code{True} & \code{False} \\
      \tabucline \\
      \code{True} && \\
      \tabucline \\
      \code{False} && \\
    \end{tabu}
  \end{center}
\end{exercise}

\section{Numerical tests and some mathematical language}

In the lemmas in the previous section, it's awfully tiring to write
``is equal to'' every time I want to mention that two things are
equal. Instead, Python has the \code{==} operator:

\lstpy{python-equals-equals.txt}

\begin{remark}
  It's important to note that \code{a == b} \emph{tests} to see if
  \code{a} and \code{b} are equal, whereas \code{a = b} \emph{assigns}
  \code{a} to whatever \code{b} is.
\end{remark}

The analog of $\ne$ is \code{!=}:

\begin{lstlisting}[language=Python]
>>> 2 != 4
True
>>> 2 != 2
False
\end{lstlisting}

In fact, most of the mathematical operators have analogs in
Python. \Cref{tbl:math-python} might be helpful.

\begin{table}
  \centering
  \begin{tabu}{r|l}
    \textbf{Math} & \textbf{Python} \\
    \tabucline \\
    $a := b$ & \code{a = b} \\
    $a = b$ & \code{a == b} \\
    $a \ne b$ & \code{a != b} \\
    $a > b$ & \code{a > b} \\
    $a \ge b$ & \code{a >= b} \\
    $a < b$ & \code{a < b} \\
    $a \le b$ & \code{a <= b} \\
  \end{tabu}
  \caption{Mathematical statements, and their equivalent in Python}
  \label{tbl:math-python}
\end{table}

\subsection{Some number $x$ is within $\epsilon$ of $y$}

This is a problem that comes up a lot in programming. For instance, if
you're writing a shoot-em-up game, you want to be able to tell if a
bullet hit its target. It's really hard to properly detect collisions,
like you would in real life. Instead, what you do is see if the bullet
was ever sufficiently close to its target that you can say ``that's
good enough, it hit the target''.

An example that complicated won't come up for a while. However, this
is one of the CodingBat problems:

\begin{quotation}
  Given an int $n$, return \True if it is within 10 of 100 or
  200. Note: \code{abs(num)} computes the absolute value of a number.
\end{quotation}

In that case, you want to be able to tell if some number $n$ is within
$10$ of $100$, and again check if it's within $10$ of $200$. A na\"ive
approach is this:

\begin{enumerate}
\item check if $n \ge 90$, and $n \le 110$, then
\item check if $n \ge 190$, and $n \le 210$.
\end{enumerate}

That quickly becomes cumbersome to read. Instead, take a page out of a
topology textbook: the distance between two points $x$ and $y$ is
equal to $\abs{x - y}$. Note: here, $\abs{x}$ denotes the
``magnitude,'' or ``norm'' of $x$. With ordinary numbers, $\abs{x}$ is
just the absolute value:

\begin{definition}
  For a real number $x$,

  \begin{equation}
    \label{eq:abs-val}
    \abs{x} = \left\{
      \begin{split}
        x & \text{ if } x \ge 0 \\
        -x & \text{ if } x < 0
      \end{split}
    \right.
  \end{equation}
\end{definition}

Note this corollary

\begin{corollary}
  For all real numbers $x$, $$\norm{x} = \norm{-x}.$$
\end{corollary}

This implies

\begin{corollary}
  For all real numbers $x$ and $y$, $$\norm{x - y} = \norm{y - x}.$$
\end{corollary}

Therefore, \ctext{$\norm{100 - x} \le 10$ if and only if
  $90 \le x \le 110$.}  Likewise, \ctext{$\norm{200 - x} \le 10,$ if
  and only if $190 \le x \le 210.$} Therefore, in this case, you can
simplify the problem by combining these with \code{or}:

\begin{enumerate}
\item check if $\norm{x - 100} \le 10$, using \code{abs()}
\item check if $\norm{x - 200} \le 10$, using \code{abs()}
\end{enumerate}

I'm starting to sound less like a normal person, and more like a
mathematician. Therefore, there are two routes:

\begin{enumerate}
\item ``dumb down'' my language so non-mathematicians can understand
  it, or
\item teach you what all this terminology is.
\end{enumerate}

I vote for \#2, because the mathematical methods of thinking are very
helpful. The next chapter will go over much of the mathematical
terminology that you'll need. For now, I'll give you the bare minimum
to do some of the CodingBat problems.

\subsection{Exercises}

These are the problems I could find that use the methods of this
section, but not those in the next section.

\begin{exercise}
  Use \cref{tbl:math-python} and the methods described in this section
  to do \url{http://codingbat.com/prob/p124676}. (\code{near_hundred})
\end{exercise}
\cbex{http://codingbat.com/prob/p124984}{makes10}.
\cbex{http://codingbat.com/prob/p166884}{parrot_trouble}

\section{If/then/Else}

\subsection{If/then/else}

Now that you understand what Booleans are, and how to perform simple
numerical tests, we need to be able to instruct Python to

\begin{enumerate}
\item given a Boolean \code{b};
\item if \code{b} is true, do something;
\item if \code{b} is false, do something else.
\end{enumerate}

In Python, this is achieved via the \code{if} clause. The general
syntax is

\begin{lstlisting}[language=Python]
if b:
    # do something here
else:
    # do something different
\end{lstlisting}

The \code{#} represents a comment, which is ignored.

\begin{remark}
  The above is not valid code, because you'd need actual code inside
  the \code{if} and \code{else} blocks, rather than just a comment.
\end{remark}

\begin{remark}
  Please note the trailing colons after \code{if b} and
  \code{else}. It's really easy to forget them.
\end{remark}

\paragraph{Indentation.}

Nobody can agree on good indentation practices. However, Python has a
very extensive style guide, called Python Enhancement Protocol 8, or
PEP 8: \url{https://www.python.org/dev/peps/pep-0008/}.

Here are the typical spats:

\begin{enumerate}
\item How many spaces per indentation level? It's usually either 2 or
  4, but sometimes 8.
\item Should you use spaces or tabulator characters? Nowadays, spaces
  are preferred, because everyone agrees how wide a space is. Nobody
  can agree how wide a tab is (see \#1).
\end{enumerate}

PEP 8 states the following:

\begin{quotation}
  Use 4 spaces per indentation level. 

  \dots

  Spaces are the preferred indentation method.

  Tabs should be used solely to remain consistent with code that is already indented with tabs.

  Python 3 disallows mixing the use of tabs and spaces for indentation.
\end{quotation}

Almost everyone agrees that mixing tabs and spaces is evil. This book
will follow PEP 8's suggestion and use 4 spaces.

\paragraph{Back to work.}

Let's look at a simple program that takes in two numbers from the
user, and checks if their sum is $20$.

\pyfile{code/sum_is_twenty.py}

Here's an example input:

\lstin{sum_is_twenty.txt}

Often times, if the first condition fails to be true, you'll want to
test for other conditions inside the \code{else} block:

\begin{lstlisting}[language=Python]
if b1:
    # do something here
else:
    # more conditions
    if b2:
        # do something here
    else:
        # more conditions
      if b3:
          # do something here
      else:
          # ...
\end{lstlisting}

This situation is rather common, so instead we have \code{elif}:

\begin{lstlisting}[language=Python]
if b1:
    # do something here
elif b2:
    # do something here
elif b3:
    # do something here
# ...
else:
    # finally, if everything fails, here
\end{lstlisting}

I'll do one of the CodingBat problems, and leave the rest to you.

\begin{example}
  (This is \url{http://codingbat.com/prob/p162058}).

  \begin{quotation}
    Given 2 int values, return True if one is negative and one is
    positive. Except if the parameter "negative" is True, then return
    True only if both are negative.
  \end{quotation}

  \begin{solution}
    That word salad might be tough to parse. I'll try to break it down
    into a nice pretty tree:

    Given two numbers $a$ and $b$

    \begin{enumerate}
    \item If \code{negative} is \True ,

      then, return \True if both $a, b < 0$.
    \item Else,

      return \True\ if $a < 0 < b$ or $b < 0 < a$ (i.e. one is
      positive and the other is negative).
    \end{enumerate}

    There's a handy mathematical trick, here, which is, for two
    numbers $a$ and $b$:

    \begin{align*}
      ab > 0 & \text{ if $a$ and $b$ have the same sign;} \\
      ab < 0 & \text{ if $a$ and $b$ have opposite signs;} \\
      ab = 0 & \text{ if at least one of $a$ and $b$ is zero.}
    \end{align*}

    Therefore, to test if the two numbers have opposite signs, we
    simply test for $ab < 0$.

    Here's my solution:

    \pyfile{code/codingbat/warmup1/pos_neg.py}

    It gives the correct result (see \cref{fig:pos_neg}).
  \end{solution}
\end{example}

\begin{figure}[h]
  \centering
  \answergraph{images/pos_neg.png}
  \caption{Result of my solution for \code{pos_neg()}.}
  \label{fig:pos_neg}
\end{figure}

\subsection{Exercises}

\begin{exercise}
  Construct two truth tables to convince yourself that these two
  things are the same.

  \lstpy{if-x-return-true.py}

  \begin{center}
    \begin{tabu}{r|c|c}
      \code{if/else} & \code{True} & \code{False} \\
      \tabucline \\
      \True && \\
      \tabucline \\
      \False && \\
    \end{tabu}
  \end{center}

  \lstpy{return-some-boolean.py}

  \begin{center}
    \begin{tabu}{r|c|c}
      \code{return} & \code{True} & \code{False} \\
      \tabucline \\
      \True && \\
      \tabucline \\
      \False && \\
    \end{tabu}
  \end{center}

  I put this exercise first to suggest a way to eliminate a common
  (redundant) idiom in your code, and make your code a bit cleaner.
\end{exercise}

\begin{exercise}
  Here's another common redundant idiom:

  Given a Boolean \code{x}, write out a truth table to convince
  yourself that \code{x} always has the same value as \code{x ==
    True}.

  \begin{center}
    \begin{tabu}{r|l}
      \code{x} & \code{x == True} \\
      \tabucline \\
      \True & \\
      \False & \\
    \end{tabu}
  \end{center}

  Therefore, you should never write something like \code{if x ==
    True}. Instead replace it with just \code{if x}.
\end{exercise}

\begin{exercise}
  Use the \code{abs()} function to do
  \url{http://codingbat.com/prob/p197466}. (\code{diff21})
\end{exercise}

\begin{exercise}
  Use \cref{tbl:math-python}, as well as your reasoning ability to do
  \url{http://codingbat.com/prob/p162058}. (\code{pos_neg})
\end{exercise}

\begin{exercise}
  Write a program that takes a number (via \code{input()}), and prints
  out whether or not it's greater than $34$.
\end{exercise}

\cbex{http://codingbat.com/prob/p141905}{sum_double}

\begin{exercise}
  Do all of Logic-1: \url{http://codingbat.com/python/Logic-1}.
\end{exercise}

\section{Lists and Strings}

The next important type is a \term{list}. Informally, you can think of
a string as a list of characters.\footnote{This is a rather
  inefficient implementation, but it's a decent way to think about
  strings. Nonetheless, it is used in a few languages.}

The syntax for lists is to surround them in square brackets, with the
elements separated by commas:

\lstin{lists-01.txt}

To get the $i$th element in a list $x$, use \code{x[i]}:

\lstin{lists-02.txt}

\xtb{Important:} list indices start at $0$. This is the case in most
languages.

In Python, you can use negative indices, starting with $-1$, to pick
elements from the end of the list:

\lstin{lists-03.txt}

To get the $i$th element, up to (but not including) the $j$th element,
with $i < j$, use \code{x[i:j]}:

\lstin{lists-04.txt}

This left-inclusive behavior is kind of confusing, but there's nothing
I can really do about it. If you reference \code{x[i:j]} with
$i \ge j$, then you get an empty list.

To get the length of a list (or a string), use \code{len()}:

\lstin{lists-05.txt}

You can add two lists together using \code{+}:

\lstin{lists-06.txt}

You can (on the right) multiply a list by an integer to repeat the list:

\lstin{lists-07.txt}

Proper lists (i.e. not strings) do not need to contain elements of the
same type:

\lstin{lists-08.txt}

To test if a value is an element of the list, use \code{in}:

\lstin{lists-09.txt}

\subsection{Exercises}

You're now ready to do most of the level-1 CodingBat exercises.

\cbex{http://codingbat.com/prob/p189441}{not_string}
\cbex{http://codingbat.com/prob/p149524}{missing_char}
\cbex{http://codingbat.com/prob/p153599}{front_back}
\cbex{http://codingbat.com/prob/p193507}{string_times}
\cbex{http://codingbat.com/prob/p147920}{front3}
\cbex{http://codingbat.com/prob/p165097}{front_times}
\begin{exercise}
  Do all of List-1: \url{http://codingbat.com/python/List-1}.
\end{exercise}
\begin{exercise}
  Do all of String-1: \url{http://codingbat.com/python/String-1}.
\end{exercise}

\section{Loops}

Yay! We're on the last topic in the chapter! After this, the problem
we initially set out to solve will be easy.

There are two types of loops:

\begin{enumerate}
\item \code{while}-loops, and
\item \code{for}-loops.
\end{enumerate}

\code{while}-loops are simpler, and every \code{for}-loop can be
expressed as a \code{while}-loop.

While loops are of the format:

\lstpy{while-b.py}

where \code{b} is a Boolean. You can easily make an infinite loop
with:

\lstpy{while-true.py}

However, you should never do that. Typically, what you're doing is
repeating an action until \code{b} becomes \False, then you move
on. For instance:

\lstpy{while-i-less-than-10.py}

What do you think the output of that program will be? Try it and see!

That loop can be much more easily expressed as a
\code{for}-loop. Before I go much further, it's worth mentioning the
\code{range()} function:

\lstpy{for-i-in-range-10.py}

That program has the same output as the code with the
\code{while}-loop.

\subsection{Generating a list}

A common problem is to generate a list of things, using a loop. For
instance, to generate the even positive integers less than (or equal
to) $20$:

\lstpy{even-ints-lt-20.py}

Try running it, and toying with the program, to try to figure out why
it works the way it does.

Python has some syntax, borrowed from mathematics, that makes
generating lists a bit more pleasant. That's called a
\term{comprehension}. In mathematics, $\Ns$ is the set of positive
integers. So, we could write
$$\scomp{2x \in \Ns}{x \in \Ns \text{ and } x \le 10}$$ to mean ``the
set of all numbers in $\Ns$ of the form $2x$, where $x$ is in $\Ns$
and $x$ is less than or equal to $10$.'' In Python, this is

\begin{lstlisting}[language=Python]
>>> [2*x for x in range(1, 11)]
[2, 4, 6, 8, 10, 12, 14, 16, 18, 20]
\end{lstlisting}

Haskell also has list comprehensions, which have a little bit more
similarity to the mathematical notation:

\begin{lstlisting}[language=Haskell]
Prelude> [2*x | x <- [1..10]]
[2,4,6,8,10,12,14,16,18,20]
\end{lstlisting}

I'll go through one example, then give you some exercises

\begin{example}
  (From \url{http://codingbat.com/prob/p166170}).

  Given an array (list) of integers, return the number of 9's in the
  list.

  \begin{solution}
    There are several ways to do this, but here are the two I thought
    of:

    \begin{enumerate}
    \item Create an integer, call it $z$, set it equal to $0$. Loop
      through the list. For each element $x$ in the list, if $x = 9$,
      then increment $z$. Else, continue. The code would look
      something like this:

      \pyfile{code/codingbat/warmup2/array_count9.py}

      That approach works (see \cref{fig:array_count9-1}).

    \item However, the Haskellish way to do it, which is technically
      less efficient, but is much clearer, is this:

      \pyfile{code/codingbat/warmup2/array_count9-2.py}

      This also works (see \cref{fig:array_count9-2})
    \end{enumerate}
  \end{solution}
\end{example}

\begin{figure}[h]
  \centering
  \answergraph{images/array_count9-1.png}
  \caption{Approach 1 to \code{array_count9}.}
  \label{fig:array_count9-1}
\end{figure}

\begin{figure}[h]
  \centering
  \answergraph{images/array_count9-2.png}
  \caption{Approach 2 to \code{array_count9}.}
  \label{fig:array_count9-2}
\end{figure}

\subsection{Exercises}

\cbex{http://codingbat.com/prob/p118366}{string_splosion}
\cbex{http://codingbat.com/prob/p110166}{array_front9}
\cbex{http://codingbat.com/prob/p170842}{double_char}

\printbibliography

\end{document}
